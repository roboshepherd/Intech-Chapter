%\documentclass[draft]{intech}
\documentclass{intech}
%\usepackage{your_package}	%if you need custom package
%\usepackage[notquote]{hanging}
\usepackage[nolist]{acronym}
%\usepackage[colorlinks=false]{hyperref}
%\usepackage{graphicx,hypernat,subfig}
%\usepackage{color}
\usepackage[nolist]{acronym}
\usepackage{multirow}
\usepackage{booktabs}
\usepackage{rotating}
\usepackage{listings,program,threeparttable}
\usepackage{algorithmicx,algpseudocode}

% * CHAPTER NUMBER * BOOK NAME * AUTHOR(S) NAME *****************************
\setcounter{chapter}{0} % It will be set by technical editor.

\booktitle{Will-be-set-by-IN-TECH}%

\chaptertitle{Bio-inspired Communication for Self-regulated
Multi-robot Systems} % You know your chapter title?

\authors{Md Omar Faruque Sarker and Torbj{\o}rn S. Dahl}
\affiliation{University of Wales, Newport}
\country{United Kingdom}


% END * CHAPTER NUMBER * BOOK NAME * AUTHOR(S) NAME *************************

\begin{document}

\maketitle

%%
\begin{acronym}
%==== A B C D ====
\acro{AFM}{attractive field model}
\acro{AGV}{automated guided vehicle}
\acro{APCD}{average production completion delay}
\acro{APMW}{average pending maintenance work-load}
%\acro{AS}{active space}
\acro{BMS}{biology-inspired manufacturing system}
\acro{CCD}{charge-coupled device}
%\acro{CCM}{centralized communication mode}
\acro{DEM}{data and event management}
\acro{DOL}{division of labour}
%==== E F G H ====
%\acro{EPSRC}{Engineering and Physical Sciences Research Council}
\acro{GigE}{Gigabyte Ethernet}
\acro{GIL}{Global Interpreter Lock}
%\acro{GPS}{global positioning system}	
\acro{GSNC}{global sensing - no communication}
%\acro{GUI}{graphical user interface}
\acro{HEAD}{hybrid event-driven architecture on D-Bus}
%====  I J K L ====
%\acro{IF}{independent founders}
%\acro{INS}{indoor navigation system}
\acro{IPC}{inter-process communication}
%\acro{IR}{infrared}
%\acro{LCM}{local communication mode}
\acro{LSLC}{local sensing - local communication}
%==== M N O P ====
\acro{MOM}{maintenance only mode}
\acro{MRS}{multi-robot system}
\acro{MRTA}{multi-robot task allocation}
\acro{P2P}{peer-to-peer}
%\acro{PF}{potential field}
\acro{PMM}{production and maintenance mode}
%=====  Q R S T ====
\acro{RCC}{robot-controller client}
%\acro{RW}{random walk}
\acro{SDK}{software-development kit}
%\acro{SF}{swarm founders}
\acro{SHM}{shared memory}
%\acro{SI}{swam intelligence}
%\acro{SO}{self-organization}
%\acro{SR}{self-regulation}
%\acro{SRS}{swarm robotic system}
\acro{TPS}{task perception server}
%==== U V W X ====
%==== Y Z ==== 
\end{acronym}
%%

\section{Introduction}
In recent years, the study of biological social insects and other animals reveals us that simple individuals of these self-organized  societies can solve various complex and large everyday-problems with a few behavioural rules \citep{Camazine+2001}. In these self-organized systems, an individual agent may have limited cognitive, sensing and communication capabilities. But they are collectively capable of solving complex and large problems, e.g. coordinated nest construction of honey-bees, collective defence of school fishes from a predator attack, ordered homing of bats.  Since the discovery of these collective behavioural patterns of self-organized societies, scientists observed modulation or adaptation of behaviours in the individual level \citep{Garnier+2007}. One of the most notable such self-regulatory processes in biological social systems is the \textit{division of labour} (DOL) \citep{Sendova-Franks+1999} by which a larger task is divided into a number of small subtasks and each subtask is performed by a separate individual or a group of individuals. 

From the study of social insects and other biological societies, we can find that two major metrics of DOL have been established in literature.
\begin{enumerate}
\item[\textbf{Task-specialization.}]
%\newline
{\em Task-specialization} is an integral part of DOL where a worker usually does not perform all tasks, but rather specializes in a set of tasks, according to its morphology, age, or chance \citep{Bonabeau+1999}. This DOL among nest-mates, whereby different activities are performed simultaneously by groups of specialized individuals, is believed to be more efficient than if tasks are performed sequentially by unspecialised individuals.
%
\item[\textbf{Plasticity.}]
DOL is also characterized by {\em plasticity} which means that the removal of one class of workers is quickly compensated for by other workers. Thus distributions of workers among different concurrent tasks keep changing according to the external (environmental) and internal conditions of a colony \citep{Garnier+2007}. 
\end{enumerate}

In artificial social systems, like multi-agent or multi-robot system, the term ``division of labour'' is often found synonymous to ``task-allocation'' \citep{Shen+2001}. In robotics, this is called \textit{multi-robot task allocation} (MRTA) which is generally identified as the question of assigning tasks in an appropriate time to the appropriate robots considering the changes of the environment and/or the performance of other team members \citep{Gerkey+2004}. In this chapter, we have presented this issue of DOL as the representative self-regulatory process in both biological and artificial social systems. We have we have used the terms DOL and MRTA (or simply, task-allocation) interchangeably. 

The complexities of the distributed MRTA problem arise from the fact that there is no central planner or coordinator for task assignments, and in a large multi-robot systems, generally robots have limited capabilities to sense, to communicate and to interact locally.

Traditionally, task allocation in a multi-agent systems has been divided into the following two major categories.
\begin{enumerate}
\item[\textbf{Predefined task-allocation.}]
%\newline
Early research on predefined task-allocation was dominated by intentional coordination \citep{Parker2008}, use of dynamic role assignment \citep{Chaimowicz2002} and market-based bidding approach \citep{Dias+2006}. Under these approaches, robots use direct task-allocation method, often to communicate with group members for negotiating on tasks. These approaches are intuitive, comparatively straight forward to design and implement and can be analysed formally. However, these approaches typically works well only when the number of robots are small ($\leq 10$) \citep{Lerman+2006}.
%
\item[\textbf{Bio-inspired self-organized task-allocation.}]
This approach relies on the emergent group behaviours, such as emergent cooperation \citep{Kube+1993}, adaptation rules \citep{Liu+2007} etc. They are more robust and scalable to large team sizes. However, most of the robotic researchers found that self-organized task-allocation approach is difficult to design, to analyse (formally) and to implement in real robots. 
\end{enumerate}

Within the context of the Engineering and Physical Sciences Research Council (EPSRC) project, ``Defying the Rules: How Self-regulatory Systems Work'', we have proposed to solve the above mentioned self-regulated DOL problem in an alternate way \citep{Arcaute+2008}. Our approach is inspired from the studies of emergence of task-allocation in both biological and human social systems in which we have found that a large number of species grow, evolve and generally continue functioning well by the virtue of their individual self-regulatory DOL systems. It is interesting to note that in self-regulated societies, task-allocation has been accomplished years after years without a central authority or an explicit planning and coordinating element. Direct and indirect communication strategies are used to exchange information among individuals \citep{Camazine+2001}. 

From our multi-disciplinary studies of various self-regulated systems, we have proposed four generic rules to explain self-regulation in those social systems. These four rules are: \textit{continuous flow of information}, \textit{concurrency}, \textit{learning} and \textit{forgetting}, all of them will be explained later. Primarily these rules deal with the issue of deriving local control laws for regulating an individual's task-allocation behaviour that can facilitate the DOL in the entire group. In order employ these rules in the individual level, we have developed a formal model of self-regulated DOL, called the attractive field model (AFM). AFM provides an abstract framework for self-regulatory DOL in social systems. 

In biological social systems, communications among the group members, as well as sensing the task-in-progress, are two key components of self-organized DOL. In robotics, existing self-organized task-allocation methods rely heavily upon local sensing and local communication of individuals for achieving self-organized task-allocation. However, AFM differs significantly in this point by avoiding the strong dependence on the local communications and interactions found in many existing approaches to MRTA. AFM requires a system-wide continuous flow of information about tasks, agent states etc. but this can be achieved by using both centralized and decentralized communication modes under explicit and implicit communication strategies. 

In order to enable continuous flow of information in our multi-robot system, we have implemented two types of sensing and communication strategies inspired by the self-regulated DOL found in two types of social wasps: {\em polistes} and {\em polybia} \citep{Jeanne1999}. Depending on the group size, these species follow different strategies for communication and sensing of tasks. Polistes wasps are called the {\em independent founders} in which reproductive females establish colonies alone or in small groups (in the order of $10^2$), but independent of any sterile workers. On the other hand, polybia wasps are called the {\em swarm founders} where a swarm of workers and queens initiate colonies consisting of several hundreds to millions of individuals.

The most notable difference in the organization of work of these two social wasps is: independent founders do not rely on any cooperative task performance while swarm founders interact with each-other locally to accomplish their tasks. The work mode of independent founders can be considered as {\em global sensing - no communication (GSNC)} where the individuals sense the task requirements throughout a small colony and do these tasks without communicating with each other. On the other hand, the work mode of swarm founders can be treated as {\em local sensing - local communication (LSLC)} where the individuals can only sense tasks locally due to large colony-size and they can communicate locally to exchange information, e.g. task-requirements (although their exact mechanism is unknown). In this chapter, we have used these two sensing and communication strategies to compare the performance of the self-regulated DOL of our robots under AFM.

%==============================================================
\section{The Attractive Field Model (AFM)}
\label{afm}
\subsection{Generic framework}
\label{afm:framework}
Inspired from the DOL in ants, humans and robots,  we  have proposed the following four {\em generic rules of self-regulation}.

\textbf{Rule 1: Continuous flow of information.} Self-regulatory social systems establish the continuous minimum flow of information over the period of time when self-regulation can be defined. This should help to maintain at least two states of an agent: 1) receiving information about task(s) and 2) ignoring information or doing no task. The up-to-date information should reflect  the changes of the system i.e. it should encode the necessary feedback for the agents. Thus, this property will act as the basis of the state switching of agents, between these two minimum states or, among multiple states e.g. in case of multiple tasks or many sub-states of a single task.

At the individual level, information is processed differently by each individual, and is certainly not constant nor continuous. The time scale at the individual level is very small compared to the system level's time scale.We can approximate the propagation of information at this macro time scale as the continuous flow of information.  In the model, emphasize is given to whether the information is used e.g. stimulation to perform a task, or unused e.g. random walk.

\textbf{Rule 2: Sensitization}. Self-regulatory social systems allow the differentiation in the use of  (or access to) information, e.g. through sensitization or learning of some tasks. This differentiation is regulated by the characteristics of the system, e.g. the ability of the agents to learn tasks that are repeatedly performed.

\textbf{Rule 3: Concurrence.} Self-regulatory social systems include concurrent access to information from different spatial locations with certain preferences. This preference is not fixed and can change with the dynamics of the system. 

\textbf{Rule 4: Forgetting.} Self-regulatory social systems include forgetting, e.g. the ability of the agents to diminish information over time, if not used. The system determines the amount of information being released, and this may change over time. For example, specialists might have to attend an emergency situation and switch tasks that contributes to the forgetting of old task experiences. This is considered as crucial to allow flexibility in the system.
%%
\begin{figure}
\centering
\includegraphics[height=5cm]{./images/AFM-Diag2.eps}
\caption{The attractive filed model (AFM)}
\label{fig:afm} % Give a unique label
\end{figure}

Having this general framework of self-regulatory social systems, we can now formalize AFM that will describe the properties of individual  agents and the system as a whole. In terms of networks, the model is a bipartite network, i.e. there are two different types of nodes. One set of nodes describes the sources of the attractive fields and the other set describes the agents. Links only exist between different types of nodes and they encode the flow of information so that, even if there is no direct link between two agents, their interaction is taken into account in the information flow. This is an instance of {\em weak} interaction. The strength of the field primarily depends upon the distance between the task and the agent. This relationship is represented using weighted links. Besides, there is a permanent field that represents the {\em no-task} or  option for ignoring information. The model can be mapped to a network as shown in Fig. \ref{fig:afm}. The correspondence is given below:
\begin{enumerate}
\item Source nodes (o) are tasks that can be divided between a number of agents.
\item Agent nodes (x) an be ants, human,  robots etc.
\item The attractive fields correspond to stimuli to perform a task, and these are given by the black solid lines.
\item When an agent performs a task, the link becomes different, and this is denoted in the figure by a dashed line. Agents linked to a source by a red line are the agents currently doing that task. 
\item The field of ignoring the information (w) corresponds to the stimulus to random walk, i.e. the no-task option, and this is denoted by the dotted lines in the graph. 
\item Each of the links is weighted. The value of this weight describes the strength of the stimulus that the agent experiences. In a spatial representation of the model, it is easy to see that the strength of the field depends on the physical distance of the agent to the source. Moreover, the strength can be increased through sensitisation of the agent via experience (learning). This distance is not depicted in the network, it is represented through the weights of the links. In the figure of the network (Fig. \ref{fig:afm}), the nodes have arbitrary places. Note that even though the distance is physical in this case, the distance in the model applied to other systems, needs not to be physical. It can represent the accessibility to the information, the time the information takes to reach the receiver, etc. 
\end{enumerate}
In summary, from the above diagram of the network, we can see that each of the agents is connected to each of the fields. This means that even if an agent is currently involved in a task, the probability that it stops doing it in order to pursue a different task, or to random walk, is always non-zero. The weighted links express the probability of an agent to be attracted to each of the fields.
%---------------------------------------------------------------
\subsection{Interpretation of AFM in multi-robot systems}
\label{afm:mrs-interpretation}
The interpretation of AFM in a multi-robot system follows the above mentioned generic interpretation. Let us consider a multi-tasking environment, where $N$ number of autonomous mobile robots are required to attend $J$ number of tasks spread over a fixed area $A$. 

Let a task $j$ has an associated task-urgency $\phi_j$ indicating its relative importance over time.
If a robot attends a task $j$ in the $x^{th}$ time-step, the value of $\phi_j$ will decrease by an amount $\delta_{\phi_{INC}}$ in the $(x+1)^{th}$ time-step.
On the other hand, if a task has not been served by any robot in the $x^{th}$ time-step, $\phi_j$ will increase by another amount  $\delta_{\phi_{DEC}}$  in $(x+1)^{th}$ time-step. Thus
%-- Phi update
urgency of a task is updated by the following rules.
\begin{equation}
 If\hspace*{0.15cm}the\hspace*{0.15cm} task\hspace*{0.15cm}is\hspace*{0.15cm}not\hspace*{0.15cm}being\hspace*{0.15cm} done:\hspace*{0.15cm}  \phi_j \rightarrow   \phi_j \hspace*{0.15cm} + \delta_{\phi_{INC}}
\label{eqn:delta-phi1}
\end{equation}
%%
\begin{equation}
 If\hspace*{0.15cm}the \hspace*{0.15cm}task\hspace*{0.15cm}is\hspace*{0.15cm}being\hspace*{0.15cm}done:\hspace*{0.15cm}  \phi_j \rightarrow   \phi_j \hspace*{0.15cm} - n\hspace*{0.10cm}\delta_{\phi_{DEC}}
\label{eqn:delta-phi2}
\end{equation}
Eq. \ref{eqn:delta-phi1} refers to a case where no robot attend to task $j$ and Eq. \ref{eqn:delta-phi2} refers to another case where $n$ robots are concurrently performing the task $j$.

In order to complete a task $j$, a robot $r_i$ needs to be within a fixed boundary $D_{j}$. If a robot completes a task $j$ it learns about it and this will influence $r_i$'s likelihood of selecting that task in future, say through increasing  its sensitization to $j$ by a small amount, $k_{INC}$. Here, the variable affinity of a robot $r_i$ to task $j$ is called as its {\em sensitization} $k^{i}_{j}$. If a robot $i$ does not do a task $j$ for some time, it forgets about $j$ and $k^i_j$ is decreased, by another small amount, say $k_{DEC}$ .
Thus a robot's task-sensitization is updated by following these rules.
\begin{equation}
 If\hspace*{0.15cm}task\hspace*{0.15cm}is\hspace*{0.15cm}done:\hspace*{0.15cm}  k^i_j \rightarrow   k^i_j \hspace*{0.15cm} + \hspace*{0.15cm} k_{INC}
\label{eqn:k-inc}
\end{equation}
\begin{equation}
 If\hspace*{0.15cm}task\hspace*{0.15cm}is\hspace*{0.15cm}not\hspace*{0.15cm}done:\hspace*{0.15cm}  k^i_j \rightarrow   k^i_j \hspace*{0.15cm} - \hspace*{0.15cm} k_{DEC}
\label{eqn:k-dec}
\end{equation}

According to AFM, all robots will establish attractive fields to all tasks due to the presence of a system-wide continuous flow of information. The strength of these attractive fields will vary according to the dynamic distances between the robots and tasks, task-urgencies and corresponding sensitizations of robots. Simplifying the generic implementation of AFM from \citet{Arcaute+2008}, we can formally encode this stimuli of attractive field as follows.
%% S
\begin{equation}
S_{j}^{i} = tanh\{\frac{k_{j}^{i}}{d_{ij}+\delta } \phi _{j}\}
\label{eqn:afm1}
\end{equation}
%--P(Task)
\begin{equation}
S^{i}_{RW} = tanh \left \{ 1 -  \frac{ \sum_{j=1}^{J} S^{i}_{j}}{J + 1} \right \}
\label{eqn:afm2}
\end{equation}
%-- P(RW)
\begin{equation}
P_{j}^{i} = \frac{S_{j}^{i}}{\sum_{j=0}^{J} S_{j}^{i}} \hspace*{0.25cm}where,\hspace*{0.25cm}S^{i}_{0} = S^{i}_{RW}   
\label{eqn:afm3}
\end{equation}

Eq. \ref{eqn:afm1} states that the stimuli of a robot $r_i$ to a particular task $j$, $S^{i}_{j}$ depends on $r_i$'s spatial distance to $j$ ($d_{ij}$), level of sensitization to $j$ ($k_{j}^{i}$), and perceived urgency of that task ($\phi _{j}$). In  Eq. \ref{eqn:afm1}, we have used a very small constant value $\delta$, called {\em delta distance}, to avoid division by zero, in the case when a robot has reached to a task. Since $S^{i}_{j}$ is a probability function, it is chosen as a $tanh$ in order to keep the values between 0 and 1. Eq. \ref{eqn:afm2} suggests us how we can estimate the stimuli of random walk or no-task option. This stimuli of random walk depends on the sum of stimulus of $J$ real tasks. Here, random-walk is also considered as a task. Thus the total number of tasks become $J+1$. The probability of selecting each task has been determined by a probabilistic method outlined in Eq. \ref{eqn:afm3} which states that the probability of choosing a task $j$ by robot $r_i$ is directly proportional to its calculated stimuli $ S^i_j$. 
%================================================================
\section{Communication in biological social systems}
\label{bio-comm}
Communication plays a central role in self-regulated DOL of biological social systems.In this section, communication among  social insects are briefly reviewed.
%
\subsection{Purposes, modalities and ranges}
Communication in biological societies serves many closely related social purposes. Most P2P communication include: recruitment to a new food source or nest site, exchange of food particles, recognition of individuals, simple attraction, grooming, sexual communication etc. In addition to that colony-level broadcast communication include: alarm signal, territorial and home range signals and nest markers, communication for achieving certain group effect such as, facilitating or inhibiting  a group activity \citep{Holldobler1990}.
\begin{table}
\caption{Common communication modalities in biological social systems}
\label{table:bio-comm-modalities}
\begin{center}
\begin{threeparttable}
\begin{tabular}{|l|l|l|}
\hline \textbf{Modality} & \textbf{Range} & \textbf{Information type}\\
\hline Sound & Long\tnote{a} & Advertising about food  source,  danger etc. \\                                                                                                                                               
\hline Vision & Short\tnote{b}  & Private, e.g. courtship display. \\
\hline Chemical  & Short/long & Various messages, e.g. food location, alarm etc.\\
\hline Tactile & Short & Qualitative info, e.g. quality of flower,\\ & & peer identification etc.\\
\hline Electric & Short/long & Mostly advertising types, e.g. aggression messages.\\
\hline
\end{tabular}
\begin{tablenotes}
\item [a]Depending on the type of species, long range signals can reach from a few metres to several kilometres.
\item [b]Short range typically covers from few mm to about a metre or so.
\end{tablenotes}
\end{threeparttable}
\end{center}
\end{table}

%[Modalities and Ranges]
Biological social insects use different modalities to establish social communication, such as, sound, vision, chemical, tactile,  electric and so forth (Table \ref{table:bio-comm-modalities}).  Sound waves can travel a long distance and thus they are suitable for advertising signals. They are also best for transmitting complicated information quickly \citep{Slater1986}. Visual signals can travel more rapidly than sound, but they are limited by the physical size or line of sight of an animal. They also do not travel around obstacles. Thus they are suitable for short-distance private signals.

%% FIG. Fireflies
\begin{figure}[htp]
\centering
\subfigure[Flashing fireflies]
{
\includegraphics[width=6cm, height=4cm]{./photos/fire-flies.eps}
\hspace{0.25cm}
\includegraphics[width=6cm, height=4cm]{./photos/firefly-light-under.eps}
}
\caption{(a) Flashing lights of fireflies displaying their synchronous behaviours (b) A firefly can produce light to signal other fireflies. %\protect\newline 
From http://www.letsjapan.markmode.com, last seen on 01/06/2010.}
\label{fig:fireflies}
\end{figure}
%
In ants and some other social insects, chemical communication is predominant. Any kind of chemical substance that is used for communication between intra-species or inter-species is termed as {\em semiochemical} \citep{Holldobler1990}. A pheromone is a semiochemical, usually a type of glandular secretion, used for communication within species. One individual releases it as a signal and others respond to it after tasting or smelling. Using pheromones individuals can code quite complicated messages in smells. For example, a typical an ant colony operates with somewhere between 10 and 20 kinds of signals. Most of these are chemical in nature. If wind and other conditions are favourable,  this type of signals emitted by such a tiny species can be detected from several kilometres away. Thus chemical signals are extremely economical of their production and transmission. But they are quite slow to diffuse away. But ants and other social insects manage to create sequential and compound messages either by a graded reaction of different concentrations of same substance or by blends of signals.

Tactile communication is also widely observed in ants and other species typically by using their body antennae and forelegs. It is observed that in ants touch is primarily used  for receiving information rather than informing something. It is usually found as an invitation behaviour in worker recruitment process. When an ant intends to recruit a nest-mate for foraging or other tasks it runs upto a nest-mate and beats her body very lightly with  antennae and forelegs. The recruiter then runs to a recently laid pheromone trail or lays a new one. In this form of communication limited amount of information is exchanged. In underwater environment some fishes and other species also communicate through electric signals where their nerves and muscles work as batteries. They use continuous or intermittent pulses with  different frequencies to learn about environment and to convey their identity and aggression messages.
%%
%%%%%%%%%%%%%%%%%%%%%%%%%%%%%%%%%%%%%%%%%%
\subsection{Signal active space and locality}
%bio-comm-ants-active-space
\begin{figure}
\centering
\includegraphics[width=12cm, angle=0]
{./images/bio-comm-ants-active-space.eps}
%figure caption is below the figure
\caption{Pheromone active space observed in ants, reproduced from \protect\cite{Holldobler1990}.}
\label{fig:ants-active-space} % Give a unique label
\end{figure}
The concept of active space (AS) is widely used to describe the propagation of signals by species. In a network environment of signal emitters and receivers, active space is defined as the area encompassed by the signal during the course of transmission \citep{Mcgregor2000}. In case of long-range signals, or even in case of short-range signals, this area include several individuals where their social grouping allows them to stay in cohesion. The concept of active space is described somewhat differently in case some social insects. In case of ants, this active space is defined as a zone within which the concentration of pheromone (or any other behaviourally active chemical substances) is at or above threshold concentration \citep{Holldobler1990}. Mathematically this is denoted by a ratio:
\begin{equation}
AS = \frac{\textit{Amount of pheromone emitted (Q)}}{\textit{Threshold concentration at which the receiving ant responds (K)}}
\end{equation}
Here, Q is measured in number of molecules released in a burst or in per unit of time whereas K is measured in molecules per unit of volume. 
Fig. \ref{fig:ants-active-space} shows the use of active spaces of two species of ants: (a) {\em Atta texana} and (b) {\em Myrmicaria eumenoides}.  The former one uses two different concentrations of {\em 4-methyl-3-heptanone} to create attraction and alarm signals, whereas the latter one uses two different chemicals: {\em Beta-pinene} and {\em Limonene} to create similar kinds signals, i.e. alerting and circling.
 
The adjustment of this ratio enables individuals to gain a shorter fade-out time and permits signals to be more sharply pinpointed in time and space by the receivers. In order to transmit the location of the animal in the signal, the rate of information transfer can be increased by either lowering the rate of emission of Q or by increasing K, or both. For alarm and trail systems a lower value of this ratio is used. Thus, according to need, individuals regulate their active space by making it large or small, or by reaching their maximum radius quickly or slowly, or by enduring briefly or for a long period of time. For example, in case of alarm, recruitment and sexual communication signals where encoding the location of an individual is needed, the information in each signal increases as the logarithm of the square of distance over which the signal travels. From the precise study of pheromones it has been found that active space of alarm signal is consists of a concentric pair of hemispheres (Fig. \ref{fig:ants-active-space}). As an ant enters the outer zone, she is attracted inward toward the point source; when she next crosses into the central hemisphere she become alarmed. It is also observed that ants can release pheromones with different active spaces.

% indirect & b/c comm
\begin{figure}
\begin{minipage}[t]{0.48\linewidth}
\centering
\includegraphics[width=6cm, height=4cm, angle=0]
{./photos/ants_group_comm_bioteams_com.eps}
\caption{A group of ants following pheromone-trail. %\protect\newline 
From http://www.bioteams.com, last seen on 01/06/2010.}
\label{fig:ant-indirect} % Give a unique label
\end{minipage}
\hspace{0.5cm}
\begin{minipage}[t]{0.48\linewidth}
\centering
\includegraphics[width=6cm,height=3.5cm, angle=0]{./photos/honey-bee-waggle-dance-knol-google.eps}
\caption{ A dancing honey-bee (\protect{\em centre}) and its followers. 
%\protect\newline 
From http://knol.google.com, last seen on 01/06/2010.}
\label{fig:honey-bee-local-bc} % Give a unique label
\end{minipage}
\end{figure}
Active space has strong role in modulating the behaviours of ants. For example, when workers of {\em Acanthomyops claviger} ants produce alarm signal due to an attack by a rival or insect predator, workers sitting a few millimetres away begin to react within seconds. However, those ants sitting a few centimetres away take a minute or longer to react. In many cases, ants and other social insects exhibit modulatory communication within their active space where many individuals involve in many different tasks. For example, while retrieving the large prey, workers of {\em Aphaeonogerter} ants produce chirping sounds (known as \textit{stridulate}) along with releasing poison gland pheromones. These sounds attract more workers and keep them within the vicinity of the dead prey to protect it from their competitors. This communication amplification behaviour can increase the active space to a maximum distance of 2 meters.
\begin{table}
\caption{Common communication strategies in biological social systems}
\label{table:bio-comm-strategy}
\begin{center}
\begin{tabular}{|l|l|}
\hline 
\textbf{Communication strategy} & \textbf{Common modalities used}\\
\hline 
Indirect & Chemical and electric \\
%\hline 
P2P &  Vision and tactile\\
%\hline 
Local broadcast &  Sound, chemical and vision\\
%\hline 
Global broadcast & Sound, chemical and electric\\
\hline
\end{tabular}
\end{center}
\end{table}
%%%%%%%%%%%%%%%%%%%%%
\subsection{Common communication strategies}
\label{bg:bio-comm:strategies}
%%%
% FIG. Honey bee dance language
%\begin{figure}
%\centering
%\subfigure[Honey-bee's waggle dance]
%{
%\includegraphics[width=6cm, height=4cm]{./photos/honey-bee-round-dance.eps}
%\hspace{0.25cm}
%\includegraphics[width=6cm, height=4cm]{./photos/honey-bee-waggle-dance.eps}
%}
%\caption{Examples of local broadcast communication of honey-bees: (a) Honey-bees show waggle-dance (making figure of 8) when food is far and (b) they show round-dance without any waggle when food is closer (within about 75m of hive). From \protect\cite{Slater1986}.}
%\label{fig:honey-bee-dances}
%\end{figure}
%%
In biological social systems, we can find all different sorts of communication strategies ranging from indirect pheromone trail laying to local and global broadcast of various signals. Sec. \ref{bg:def:comm} discusses the most common four communication strategies in natural and artificial world, i.e. indirect, P2P, local and global broadcast communication strategies. Table \ref{table:bio-comm-strategy} lists the use of various communication modalities under different communication strategies. Here we give a few real examples of those strategies from biological social systems. In biological literature, the pheromone trail laying is one of the most discussed indirect communication strategy among various species of ants. Fig. \ref{fig:ant-indirect} shows a pheromone trail following of a group of foraging ants. This indirect communication strategy effectively helps ants to find a better food source among multiple sources, find shorter distance to a food source, marking nest site and move there etc. \citep{Hughes2008}. Direct P2P communication strategy is also very common among most of the biological species. Fig. \ref{fig:bees-ants-p2p-comm} shows P2P communication of ants and honey-bees. This tactile form of communication is very effective to exchange food item, flower nectar with each-other or this can be useful even in recruiting nest-mates to a new food source or nest-site.
%%%%%%%%%%%%%%%%%%%%%%
\begin{figure}
\centering
\subfigure[Two honey-bees]
{
\includegraphics[width=6cm, height=4cm]{./photos/honey-bee-p2p-hy23.eps}
\hspace{0.25cm}
\includegraphics[width=6cm, height=4cm]{./photos/ants-p2p-hy14.eps}
}
\caption{Example of P2P tactile communication: (a) Honey-bees exchange nectar samples by close contact (b) ants also exchange food or information via tactile communication. \protect\newline  From http://www.harunyahya.com/ last seen 01/05/2010.}
\label{fig:bees-ants-p2p-comm}
\end{figure}
%%
%%-----------------------------------------------------
\subsection{Roles of communication in task-allocation}
\label{bg:bio-comm:comm-role}
Communication among nest-mates and sensing of tasks are the integral parts of the self-regulated DOL process in biological social systems. They create necessary  preconditions for switching from one tasks to another or to attend dynamic urgent tasks. Suitable communication strategies favour individuals to select a better tasks. For example, \cite{Garnier+2007} reported two worker-recruitment experiments on black garden ants and honey-bees. The scout ants of {\em Lasius niger}  recruit uninformed ants to food source using a well-laid pheromone trails. {\em Apis mellifera} honey-bees also recruit nest-mates to newly discovered distant flower sources through waggle-dances. In the experiments,  poor food sources were given first to both ants and honey-bees. After some time,  rich food sources were introduced  to them. It was found that only honey-bees were able to switch from poor source to a rich source using their sophisticated dance communication.

%%
\begin{table}
\caption{Self-regulation of communication behaviours in biological social systems}
\label{table:bio-comm-task-urgency}
\begin{center}
\begin{tabular}{|l|l|l|}
\hline \textbf{Example event} & \textbf{Strategy} & \textbf{Modulation of communication}\\
&  &  \textbf{upon sensing tasks}\\
\hline Ant's alarm signal &  Global  & High concentration of pheromones\\
by pheromones & broadcast &  increase aggressive alarm-behaviours \\                                                                                                                                               
\hline Honey-bee's  & Local  &  High quality of nectar source increases \\
round dance & broadcast & dancing and foraging bees\\
\hline Ant's tandem run     & P2P & High quality of nest \\
for nest selection & &   increases traffic flow\\
\hline Ant's pheromone   & Indirect & Food source located at shorter distance\\
trail-laying to   & &  gets higher priority as less pheromone \\
food sources & & evaporates and more ants joins\\
\hline
\end{tabular}
\end{center}
\end{table}
%%%%%%%%%%%%%%%%%
Table \ref{table:bio-comm-task-urgency} presents the link between sensing the task and self-regulation of communication behaviours among ants and honey-bees. Here, we can see that communication is modulated based on the perception of  task-urgency irrespective of the communication strategy of a particular species. Under indirect communication strategy of ants, i.e. pheromone trail-laying, we can see that the principles of self-organization, e.g. positive and negative feedbacks take place due to the presence of different amount of pheromones for different time periods. Initially, food source located at shorter distance gets relatively more ants  as the ants take less time to return nest. So, more pheromone deposits can be found in this path as a result of positive feedback process.  Thus, the density of pheromones or the strength of indirect communication link reinforces ants to follow this particular trail.

%%
\begin{figure}
\centering
\includegraphics[width=6cm, angle=-90]
{./images/honey-bee-dance-stat.eps}
%figure caption is below the figure
\caption{Self-regulation in honey-bee's dance communication behaviours, produced after the results of \protect\cite{Von1967} honey-bee round-dance experiment performed on 24 August 1962.}
\label{fig:honey-bee-dance-stat}  % Give a unique label
\end{figure}
%%
Similarly, perception of task-urgency influences the P2P and broadcast communication strategies. {\em Leptothorax albipennis} ant take lees time in assessing a relatively better nest site and quickly return home to recruit its nest-mates \citep{Pratt+2002}. Here, the quality of nest directly influences its intent to make more ``tandem-runs'' or to do tactile communication with nest-mates. We have already discussed about the influences of the quality of  flower sources to honey-bee dance.  Fig. \ref{fig:honey-bee-dance-stat} shows this phenomena more vividly. It has been plotted using the data from the honey-bee round-dance experiments of \cite[p. 45]{Von1967}. In this plot, Y1 line refers to the concentration of sugar solution. This solution was kept in a bowl  to attract honey-bees and the amount of this solution was varied from $\frac{3}{16}$M to 2M (taken as 100\%). In this experiment, the variation of this control parameter influenced honey-bees' communication behaviours while producing an excellent self-regulated DOL.

In Fig. \ref{fig:honey-bee-dance-stat} Y2 line represents the number of collector bees that return home. The total number of collectors was 55 (taken as 100\%). Y3 line plots the percent of collectors displaying round dances. We can see that the fraction of dancing collectors is directly proportional to the concentration of sugar solution or the sensing of task-urgency. Similarly, the average duration of dance per bee  is plotted in Y4 line. The maximum dancing period was 23.8s (taken as 100\%). Finally, from Y5 line we can see the outcome of the round-dance communication as the number of newly recruited bees to the feeding place. The maximum number of recruited bees was 18 (taken as 100\%). So, from an overall observation, we can see that bees sense the concentration of food-source  as the task-urgency and they self-regulate their round-dance communication behaviour according to their perception of this task-urgency. Thus, this self-regulated dancing behaviour of honey-bees attracts an optimal number of inactive bees to work.

Broadcast communication is one of the classic ways to handle dynamic and urgent tasks in biological social systems. It can be commonly observed in birds, ants, bees and many other species. Table \ref{table:bio-comm-task-urgency} mentions about the alarm communication of ants. Similar to the honey-bee's dance communication, ants has a rich language of chemical communication that can produce words through blending of different glandular secretions in different concentrations. Fig. \ref{fig:ants-active-space} shows how ants can use different concentrations of chemicals to make different stimulus for other ants. From the study of ants, it is clear to us that taking defensive actions, upon sensing a danger, is one of the highest-priority tasks in an ant colony. Thus, for this highly urgent task, ants almost always use their global broadcast communication strategy through their strong chemical signals and they make sure all individuals can hear about this task.  This gives us a coherent picture of the self-regulation of biological species based on their perception of task-urgency.
\begin{figure}
\centering
\subfigure[Socail wasps]
{
\includegraphics[width=6cm, height=4cm]{./photos/Wasps_wikimedia.org_Polistes_nest_3_sjh.eps}
\hspace{0.25cm}
\includegraphics[width=6cm, height=4cm]{./photos/Polybia_occidentalis_I_JP6646_discoverlife.org.eps}
} 
\caption{Colony founding in two types of social wasps (a) {\em Polistes}  founds colony by a few queens independently (b) {\em Polybia occidentalis}  founds colony by swarms. From http://www.discoverlife.org, last seen 01/05/2010.}
\label{fig:social-wasps}
\end{figure}
%%-----------------------------------------------
\subsection{Effect of group size on communication}
\label{bg:bio-comm:group-size}
The performance of cooperative tasks in large group of individuals also depends on the communication and sensing strategies adopted by the group. As introduced in Sec. \ref{intro:comm}, from the study of social wasps,  we can find that depending on the group size, different kinds of information flow occur in different types of social wasps \citep{Jeanne1999}. Polistes independent founders (Fig. \ref {fig:social-wasps}(a)) are species in which reproductive females establish colonies alone or in small groups with about $10^2$ individuals at maturity. Polybia swarm founders (Fig. \ref {fig:social-wasps}(b)) initiates colonies by swarm of workers and queens. They have a large number of individuals, in the order of $10^6$ and 20\% of them can be queen. 
\begin{figure}[htp]
\centering
\includegraphics[width=9cm, angle=0]
{./images/jannae-fig10-info-flow-cmp.eps}
%figure caption is below the figure
\caption{Different patterns of information flow in two types of social wasps: polybia and polistes, reproduced from \protect\cite{Jeanne1999}.}
\label{fig:wasps-info-flow}  % Give a unique label
\end{figure}
%%
%\begin{figure}[htp]
%\centering
%\includegraphics[width=6cm, angle=0]
%{./images/jeanne-fig9-info-flow.eps}
%%figure caption is below the figure
%\caption{Information flow in polybia social wasps, reproduced from \protect\cite{Jeanne1999}.}
%\label{figs:sf-wasps-info-flow}  % Give a unique label
%\end{figure}
Fig. \ref{fig:wasps-info-flow} compares the occurrence of information flow among independent and swarm founders. In case of swarm founders information about nest-construction or broods food-demand can not reach to foragers directly.  Among the swarm founders for nest construction. The works of {\em pulp foragers} and {\em water foragers} depend largely on their communication with {\em builders}. On the other hand, in case of independent founders there is no such communication and sensing are present among individuals. In Sec. \ref{intro:comm} we have termed these two types communication and sensing strategies as GSNC (for independent founders) and LSLC (swarm founders).

\cite{Jeanne1999} explained the above phenomena of selecting different strategies in terms of task-specialization patterns and stochastic properties found in the group. In case of large colonies, many individuals repeatedly performs same tasks as this minimizes their interferences, although they still have a little probability to select a different task randomly. But because of the large group size, the queuing delay in inter-task switching keeps this task-switching probability very low. Thus, in swarm founders, task-specialization becomes very high among individuals. On the other hand, in small group of independent founders, specialization on a specific task is very costly, because this prevents individuals not to do other tasks whose task-urgencies can soon become very high. Thus these individuals tend to become generalist and do not communicate task information with each other.
%%
The above interesting findings on GSNC and LSLC in social wasps have been linked up with  the group productivity of wasps. \cite{Jeanne1999} reported high group productivity in case of LSLC of swarm founders. The per capita productivity was measured as the number of cells built in the nest and the weight of dry brood in grams. In case of independent founders this productivity is much lesser (max. 24 cells per queen at the time the first offspring observed) comparing to the thousands of cells produced by swarm founders.  This shows  us the direct link between high productivity of social wasps and their selection of LSLC strategy. These fascinating findings from wasp colonies have motivated us to test these communication and sensing strategies in a fairly large multi-robot system to achieve an effective self-regulated MRTA.
%==============================================================
\section{Communication in multi-robot systems}
\label{mrs-comm}
\label{bg:mrs-comm}
Communication plays an important role for any high-level interaction (e.g. cooperative or coordinative) among a multi-robot team \cite{Arkin1998}. This is not a prerequisite for the group to be functioning, but often act as a useful component of multi-robot system. The characteristics of communication in multi-robot system can be presented in terms of these issues:
\begin{itemize}
\item Rationale of communication: {\em why do the robots communicate?} 
\item Message content: {\em what do they communicate?} 
\item Communication modalities: {\em How do they communicate?} 
\item Target recipients: {\em With whom do they communicate?}
\end{itemize}
Below we have described the above issues with a focus on how communication  can lead to produce effective MRTA solutions.
%-----------
\subsection{Rationale of communication}
From three kinds of communication experiments: indirect stigmergic communication, direct robot-robot state communication, and goal communication, \citep{Balch2005}  found that in some tasks communication provided performance improvements while others did not. Most of the robotic researchers generally agree that communication in multi-robot system usually provides several major benefits listed below.
\begin{enumerate}
\item \textbf{Improved perception: }
Robots can exchange potential information (as discussed below) based on their spatial position and knowledge of past events. This, in turn, leads to improve perception over a distributed region without directly sensing it.
\item \textbf{Synchronization of actions: }
In order to perform (or stop performing) certain tasks simultaneously or in a particular order, robots need to communicate, or signal, to each other. 
\item \textbf{Enabling interactions and negotiations: }
Communication can help a lot to influence each-other in a team that, in turn, enables robots to interact and negotiate their actions effectively.
\end{enumerate}
%----------------- 
\subsection{Information content}
Although communication provides several benefits for team-work it is costly to provide communication support in terms of hardware, firmware as well as run-time energy spent in communication. So robotic researchers carefully minimize the necessary information content in communications by using suitable communication protocols and high-level abstractions. For example in foraging, grazing and consuming experiments, \cite{Balch+1994} introduced state and goal communications. In state communication, a single bit is transmitted indicating the current state of robot (e.g. 0 transmitted when robot was in {\em Wander} state and 1 transmitted when robot was in {\em Acquire} or {\em Retrieve} states). In case of goal communication, the location of task was also transmitted. Here is a brief summary of potential information contents used in communication among robots.
%%
\begin{itemize}
\item \textbf{Individual state:} ID number, battery level, task-performance statistics, e.g. number of tasks done.
\item \textbf{Goal:} Location of target task or all tasks discovered.
\item \textbf{Task-related state:} The amount of task completed, number of other robots present there etc.
\item \textbf{Environmental state:} Free and blocked paths, level of interference found, any urgent event or dangerous changes found in the environment.
\item \textbf{Intentions:} Detail plan for doing a task, sequences of selected actions etc.
\end{itemize}
%% 
 Since a multi-robot system can be comprised of robots of various computation and communication capabilities, these information contents can vary greatly based on their individual communication modules and channel capacities.
%------------------------------------------
\subsection{Communication modalities}
%\begin{figure}
%\centering
%\subfigure[ E-puck robots with table-lamps and Sniffing Khepera III]
%{\includegraphics[width=6cm, height=4cm]{./photos/distributed_table_lamp_triangle.eps}
%\hspace{0.25cm}
%\includegraphics[width=6cm, height=4cm]{./photos/odor_loc_khepera3odorprototype300x169.eps}}
%\caption{ (a) A fleet of mobile ``lighting" robots moving on a large table, such that the swarm of robots form a distributed table light and (b) Distributed odour source localization by Khephera robot equipped with volatile organic compound sensor and an anemometer (wind sensor). From http://http://disal.epfl.ch, last seen~01/06/2010.}
%\label{fig:epfl-disal}
%\end{figure}
%%
%%
Robotic researchers typically use robot's on-board wireless radio, infra-red (IR), vision and sound hardware modules for robot-robot and robot-host communication. The reduction in price of wireless radio hardware chips e.g. wifi (ad-hoc WLAN 802.11 network) or Bluetooth\footnote{www.bluetooth.com}makes it possible to use wireless radio communication widely. In-expensive IR communication module is also typically built into almost all mobile robots due to its low-cost and suitability for ambient light and obstacle detection. IR can also be used for low bandwidth communication in short-range, e.g. keen-recognition. Most robots can also produce basic sound waves and detect it with their built-in speakers and suitable configuration of on-board microphones. Although speech-recognition is not commonly found in mobile robots yet, detection of pre-recorded sound waves can be feasible.

Most of the mobile robots come with a series of LEDs, and tiny camera that can emit light signals and detect it with camera. Many robots can also detect blobs of colours and can recognize peers of other objects through the use of a color-coded markers. Some other researchers also tried to establish communication among robots solely relying on vision \citep{Kuniyoshi1994}. Although a lot of researches have been carried out to design robot skin and tactile communication system, we do not know any instance of tactile communication used in multi-robot team. In terms of chemical communication, \cite{Lochmatter+2007} showed  limited success in odour-source localization, a form of detecting chemical signals.
%-----------------------------
\subsection{Communication strategies}
\begin{figure}
\centering
\includegraphics[width=6cm, angle=0]
{./images/mrs-comm-complexities.eps}
%figure caption is below the figure
\caption{Three aspects of communication in multi-robot system}
\label{fig:mrs-comm-strategies} % Give a unique label
\end{figure}
Whatever be the communication modalities of a multi-robot system, suitable strategies are required to disseminate information in a timely manner to a target audience that maximizes the effective task-completion and minimizes delays and conflicts. A review of various communication strategies in social system has been presented in Sec. \ref{bg:def:comm}. Here, in order to discuss the complexities of communication strategies we have selected three independent scales: organization, expressiveness and range of communication. With these independent aspects, we can measure the level of complexities in communication and classify a multi-robot system according to its communication characteristics. Fig. \ref{fig:mrs-comm-strategies} outlines these scales and we have described them  below.
%%
\subsection*{Centralized and decentralized communications}
Similar to the organization of task-allocation discussed in Sec. \ref{bg:mrta:3a-taxonomy}, communication in a multi-robot system can be organized using an external/internal central entity (e.g. a server PC, or a leader robot) or, a few leader robots, or by using decentralized or local schemes where every robot has the option to communicate with every other robot of the team. From a recent study of multi-robot flocking \cite{Celikkanat+2008} have shown that a mobile robot flock can be steered toward a desired direction through externally guiding some of its members, i.e. the flock relies on multiple leaders or information repositories. Note that here task-allocation is fully decentralized i.e. each robot selects its task, but the communication structure is hybrid; robots communicate with each other and with a centralized entity.
%%--------------------------------
\subsection*{Explicit and implicit communications}
Communication in a multi-robot system can also be characterized its expressiveness or the degree of explicitness. In one extreme it can be fully implicit, e.g. stigmergic, or on the other end, it can be fully explicit where communication is done by a rich vocabulary of symbols and meanings. Researchers generally tend to stay in either end based on the robotic architecture and task-allocation mechanism used. However, both of these approaches can be tied together under any specific application. They are highlighted below.
%%
\begin{enumerate}
\item \textbf{Explicit or direct communication: }
This is also known as intentional communication. This is done purposefully by usually using suitable modality e.g. wireless radio, sound, LEDs. Because explicit communication is costly in terms of both hardware and software, robotic researchers always put extra attention to design such a system by analysing strict requirements such as communication necessity, range, content, reliability of communication channel (loss of message) etc.
%
\item \textbf{Implicit or indirect communication:} 
This is also known as indirect stigmergic communication. This is a powerful way of communication where individuals leave information in the environment. This method was adopted from the social insect behaviour, such as stigmergy of ants (leaving of small amount of pheromone or chemicals behind while moving in a trail).
\end{enumerate}
%%
%%--------------------------
\subsection*{Local and Broadcast communications}
The target recipient selection or determining the communication range or sometimes called radius of communication is an interesting issue in multi-robot system research. Researchers generally tries to maximize the information gain by using larger range. However, transmission power and communication interference among robots play a major role to limit this range. The following major instances of this strategy can be used.
\begin{itemize}
\item \textbf{Global broadcast communication:} where all robots in the team can receive the message.
\item \textbf{Local broadcast communication:} where a few robots in local neighbourhood can receive the message.
\item \textbf{Publish-subscribe communication:} where only the previously subscribed robots can receive the message.
\item \textbf{P2P communication:} where only the closest peer robot can receive the message.
\end{itemize}
%----------------------------------------------------------
\subsection*{Determination of local neighbourhood in multi-robot communication}
Most researchers in the area of swarm-robotic system, who use algorithms based on local-neighbourhood of communication, face this problem of defining the range of local neighbourhood. \cite{Agah+1995} presented that larger communication range is not always optimum for some types of tasks e.g. exploration where a large number of recipient robots decreased the performance of exploration task. \cite{Yoshida+2000} provided a design of optimal communication range of homogeneous robots based on their spatial and temporal analyses of information diffusion within the context of cooperative tasks in a manufacturing shop-floor. Spatial design tried to minimize the time for information transmission and temporal design tried to minimize the information announcing time to avoid excessive information diffusion. Eq. \ref{egn:yoshida-range} describes their optimal range $\chi_{optimal}$ as a function of information acquisition capacity of robots ($c$) and the probability of information output of a robot ($p$). Here $c$ is an integer representing the upper-limit of the number of robots that can be the target recipients at any time without the loss of information and $\chi_{optimal}$ gives the average number of robots within the output range.
\begin{equation}
\chi_{optimal} = \frac{\sqrt [c] {c!}}{p}
\label{egn:yoshida-range}
\end{equation}
%%
%%-------------------------------------------------
\subsection{Role of communication in MRTA}
Although researchers in the field of multi-robot system have been adopting various communication strategies for achieving MRTA solutions in different task domains, very few studies correlate the role of communication with the effectiveness of MRTA. This is due to the fact that researchers usually adopt a certain task-allocation method and they limit their use of communication strategy to either explicit global/local broadcast (in most predefined task-allocation researches) or  implicit/no communication (in most self-organized task-allocation researches). In the former one, communication becomes the part and parcel of the robot-robot interactions that enable them to exchange variety of information as discussed before. But in the latter one, the environment serves as a shared memory for all robots to access information or sense the current state of the environment, mostly locally. Here we have attempted to scrutinize how MRTA solutions have been affected by the variations in communication strategies.

\cite{kalra+2007} empirically studied the comparative performance of MRTA under both predefined and self-organized approaches with event-driven simulations. They found that the accuracy of information is crucial for predefined market-based approach where every robot communicated with every other robot (i.e. global broadcast). In case of unreliable link or absence of communication, threshold-based approach performed same as market-based approach, but with less computational overhead. This indicates the dependence of predefined approach on reliable communication links.However their global broadcast strategy is not feasible for large teams of real-robots. In case of varying robot's communication range,  they found that market-based approach performed well for a short communication-range where robots were able to communicate with less than a third of the total number of team-mates. Since the events are handled based-on their spatial locations only, it is not clear how this strategy  will perform in other task-domains.  
  
In order to pursue MRTA, robots can receive information from a centralised source \cite{Krieger+2000} or from their local peers \citep{Agassounon+2002}. This centralized communication system is easy to implement. It simplifies the overall design of a robot controller. However as we mentioned before, this system has disadvantage of a single point of failure and it is not scalable. The increased number of robots and tasks cause inevitable increase in communication load and transmission delay. Consequently, the overall system performance degrades. On the other hand, uncontrolled reception of information from decentralized or local sources is also not free from drawbacks. If a robot exchanges signals with all other robots, it might get the global view of the system quickly and can select an optimal or near optimal task. This can produce a great improvement in overall performance of some types of tasks e.g., in area coverage \cite{Rutishauser+2009}. But this is also not practical and scalable for a typically large multi-robot system due to the limited communication and computational capabilities of robots and limited available communication bandwidth of this type of system.

A potential alternate solution to this problem can be obtained by decreasing the number of message recipients on the basis of a local communication range. This means that robots are allowed to communicate only with those peers who are physically located within a pre-set distance. When this strategy is used for sharing task information among peers, MRTA can be more robust to the dynamic changes in the environment and energy- efficient \citep{Agassounon+2002}. Simailar to this, \cite{Pugh+2009} reported a distributed multi-robot learning scenario with two cases: 1) robots were allowed to communicate with any two other robots ({\em Model A}) and 2)  robots were allowed to communicate with all robots in a fixed radius ({\em Model B}). In simulation and real robotic experiments with 10 robots and communication ranges of 0.3 m, 1.0 m and 3.3 m, they showed that Model B performed better in intermediate communication range. However, these learning process of individual robot controller were  conducted  in static environment.  So it is not clear why intermediate communication range performs better than other ranges. 

Many robotic researchers tried to use some forms of adaptation rules in local communication to avoid saturation of the communication channel, e.g. based on robot densities in a given area.  As mentioned before,  \cite{Yoshida+2000} tried to formalize the suitable communication range based on spatial and temporal properties of information diffusion of a given communication channel. The major focus of this type of research is to measure the cost of communication based on some metrics, e.g.  transmission time and collisions with other robots, and then regulate communication strategies or ranges dynamically ranging from global broadcast to local P2P or not doing communication at all when a huge cost is involved. These ideas are attractive to maximize information gain in dynamic environment, but there is no point of doing communication if there is little or no task-requirement. Thus we find this approach, i.e. maximizing information gain,  is not always useful or necessary for effective MRTA.

\cite{Oca+2005} also acknowledged the above fact within the context of their ant-based clustering experiments. They used two simple communication strategies: 1) simple memory sharing by robots (shared memory access) and 2) shared use of environment maps (global sensing). In both of these cases, it was found that communication is only useful when some initial random clustering phase was passed. The accuracy of shared information in highly dynamic environment was poor and did not carry any significant advantage. In case of local memory sharing by robots, they showed that  sharing information within a limited number of robots produced more efficient  clusters, rather than  not sharing information at all in stigmergic communication mode. However, sharing memory in a large group is not a feasible communication strategy because of the huge latencies and interferences involved in the communication channel.
%=====================================================================
\section{Validation of AFM under centralized communication strategy}
In this section, in order to present the validation of AFM, we first describe our manufacturing shop floor scenario and then the centralized communication model along with its implementation under this scenario. Finally we present the experimental results that validates our model.
\subsection{A manufacturing shop-floor scenario}
By extending our interpretation of AFM in multi-robot system, we can set-up manufacturing shop-floor  scenario. Here, each task represents a manufacturing machine that is  capable of producing goods from raw materials, but they also require constant maintenance works for stable operations. Let $W_{j}$ be a finite number of material parts that can be loaded into a machine $j$ in the beginning of its production process and in each time-step, $\omega_{j}$ units of material parts can be processed  ($\omega_{j} \ll W_{j} $). So let $\Omega_{j}^{p}$ be the initial production workload of $j$ which is simply: $W_{j} / \omega_{j}$ unit.

We assume that all machines are identical. In each time step, each machine always requires a minimum threshold number of robots, called hereafter as {\em minimum robots per machine ($\mu$)}, to meet its constant maintenance work-load, $\Omega_{j}^{m}$ unit. However, if $\mu$ or more robots are present in a machine for production purpose, we assume that, no extra robot is required to do its maintenance work separately. These robots, along with their production jobs, can do necessary maintenance works concurrently. For the sake of simplicity, here we consider $\mu$ = 1.

Now let us fit the above production and maintenance work-loads and task performance of robots into a unit task-urgency scale. Let us divide our manufacturing operation into two subsequent stages: 1) \acfi{PMM}, and 2) \acfi{MOM}. Initially a machine starts working in PMM and does production and maintenance works concurrently. When there is no production work left, then it  enters into MOM. Fig. \ref{fig:vsp} illustrates this scenario for a single machine.

Under both modes, let $\alpha_{j}$ be the amount of workload occurs in a unit time-step if no robot serves a task and it corresponds to a fixed task-urgency $\Delta \phi_{INC}$. On the other hand, let us assume that in each time-step, a robot, $i$, can decrease a constant workload $\beta_{i}$ by doing some maintenance work along with doing any available production work. This  corresponds to a negative task urgency: $- \Delta \phi_{DEC}$. So, at the beginning of production process, task-urgency, occurred in a machine due to its production work-loads, can be encoded by Eq. \ref{eqn:task-urgency-prod-init}.
\begin{equation}
%\small
\Phi_{j, INIT}^{PMM} = \Omega_{j}^{p} \times \Delta \phi_{INC} + \phi_{j}^{m0}
\label{eqn:task-urgency-prod-init}
\end{equation}
where $\phi_{j}^{m0}$ represents the task-urgency due to any initial maintenance work-load of $j$.
Now if no robot attends to serve a machine, each time-step a constant maintenance workload of $\alpha_{j}^{m}$ will be added to $j$ and that will increase its task-urgency by $\Delta \phi_{INC}$. So, if $k$ time steps passes without any production work being done, task urgency at $k^{th}$ time-step will follow Eq. \ref{eqn:task-urgency-prod-case1}.
\begin{equation}
\Phi_{j, k}^{PMM} =\Phi_{j, INIT}^{PMM} + k \times \Delta \phi_{INC}
\label{eqn:task-urgency-prod-case1}
\end{equation}
However, if a robot attends to a machine and does some production works from it, there would be no extra maintenance work as we have assumed that $\mu$ = 1. Rather, the task-urgency on this machine will decrease by $\Delta \phi_{DEC}$ amount. If $\nu_{k}$ robots work on a machine simultaneously at time-step $k$, this decrease will be: $\nu_{k} \times \Delta \phi_{DEC}$. So in such cases, task-urgency in $(k+1)^{th}$ time-step can be represented by:
\begin{equation}
\Phi_{j, k+1}^{PMM} = \Phi_{j, k}^{PMM} - \nu_{k} \times \Delta \phi_{DEC}
\label{eqn:task-urgency-prod-case2}
\end{equation}
At a particular machine $j$, once $\Phi_{j, k}^{PMM}$ reaches to zero, we can say that there is no more production work left and this time-step $k$ can give us the {\em production completion time} of $j$, $T_{j}^{PMM}$. Average production time-steps of a shop-floor with M machines can be calculated by the following simple equation.
\begin{equation}
T_{avg}^{PMM} = \frac{1}{M} \sum_{j=1}^{M} T_{j}^{PMM} 
\label{eqn:avg-pmm}
\end{equation}
$T_{avg}^{PMM}$ can be compared with the minimum number of time-steps necessary to finish production works, $T_{min}^{PMM}$. This can only happen in an ideal case where all robots work for production without any random walking or failure. We can get $T_{min}^{PMM}$ from the total amount of work load and maximum possible inputs from all robots. If there are M machines and N robots, each machine has $\Phi_{INIT}^{PMM}$ task-urgency, and each time-step robots can decrease N $\times$ $\Delta \phi_{DEC}$ task-urgencies, then the theoretical $T_{min}^{PMM}$ can be found from the following Eq. \ref{eqn:min-pmm}.
%
%\begin{multicols}{2}
%\small
\begin{equation}
T_{min}^{PMM} = \frac{M \times \Phi_{INIT}^{PMM}}{N \times \Delta \phi_{DEC}} 
\label{eqn:min-pmm}
\end{equation}
%\vspace*{0.2cm}
\begin{equation}
\zeta_{avg}^{PMM} = \frac{T_{avg}^{PMM} - T_{min}^{PMM}}{T_{min}^{PMM}} 
\label{eqn:appd}
\end{equation}
%\end{multicols}
Thus we can define $\zeta_{avg}^{PMM}$, \acf{APCD} by following Eq. \ref{eqn:appd}:
%%
When a machine enters into MOM, only $\mu$ robots are required to do its maintenance works in each time step. So, in such cases, if no robot serves a machine, the growth of task-urgency will follow Eq. \ref{eqn:task-urgency-prod-case1}. However, if $\nu_{k}$ robots are serving this machine at a particular time-step $k^{th}$ , task-urgency at $(k+1)^{th}$ time-step can be represented by:
\begin{equation}
\Phi_{j, k+1}^{MOM} = \Phi_{j, k}^{MOM}- (\nu_{k} - \mu) \times \Delta \phi_{DEC}
\label{eqn:task-urgency-maint-case}
\end{equation}
By considering $\mu = 1$, Eq. \ref{eqn:task-urgency-maint-case} will reduces to Eq. \ref{eqn:task-urgency-prod-case2}. Here, $\Phi_{j, k+1}^{MOM}$ will correspond to the {\em pending maintenance work-load} of a particular machine at a given time. This happens due to the random task switching of robots with a no-task option (random-walking). Interestingly PMW will indicate the robustness of this system since higher PMW value will indicate the delay in attending maintenance works by robots. We can find the \acfi{APMW} per time-step per machine, $\chi_{j}^{MOM}$ (Eq. \ref{eqn:sigle-pmw}) and average PMW per machine per time-step, $\chi_{avg}^{MOM}$ (Eq. \ref{eqn:avg-pmw}).
%\begin{multicols}{2}
%\small
\begin{equation}
\chi_{j}^{MOM}= \frac{1}{K} \sum_{k=1}^{K} \Phi_{j, k}^{MOM}
\label{eqn:sigle-pmw}
\end{equation}
%\vspace*{0.2cm}
\begin{equation}
\chi_{avg}^{MOM}= \frac{1}{M} \sum_{j=1}^{M} {\chi_{j}^{MOM}}
\label{eqn:avg-pmw}
\end{equation}
%\end{multicols}
%----------------------------------------------------------
\subsection{Centralized communication model}
\begin{figure}
\centering
\includegraphics[height=5cm, angle=0]{./images/CentralizedComm.eps}
\caption{\small A centralized communication scheme} % for implementing AFM}
\label{fig:ccm} % Give a unique label
\end{figure}
%%
AFM relies upon a system-wide continuous flow of information which can be realized using any suitable communication model. A simple centralized communication scheme is outlined in Fig. \ref{fig:ccm}. In this model we have used bi-directional signal-based exchange of communication messages between a centralized \textit{task perception server} (TPS) and a \textit{robot-controller client} (RCC). The main role of TPS is to send up-to-date task-information to RCCs. This  task-information mainly contains the location and urgency of all tasks  which is used by the RCCs for running their task-allocation algorithm. The urgency value of each task is dynamically updated  by TPS after receiving the  status signals from the working robots of that particular task. Fig. \ref{fig:ccm} shows how three robots are attracted to two different tasks and their communications with TPS. Here although the robots are selecting task independently based-on the strength of their attractive fields to different tasks, they are depended on the TPS for task-information.
%%
We can characterize our communication model in terms of three fundamental issues of communication \cite{Gerkey+2001}. 
\begin{enumerate}
\item Message content: {\em what to communicate?}
\item Communication frequency: {\em when to communicate?}
\item Target message recipients: {\em with whom to communicate?}
\end{enumerate}
%%
AFM suggests the communication of task-urgencies  among robots. This communication helps the robots to gain information that can be  treated as ``global sensing''. However in this model  robots do not communicate among themselves. Hence this model can be approximated as the GSNC strategy. Since in order to run the task-allocation algorithm robot-controllers need the distance information we also include the task position information in  the message. Our centralized communication model is open to include any further information, such as time-stamp, in the message payload. In this centralized communication model the frequency of signal emission depends on several issues, e.g. the rate at which the environment is changing, the bandwidth of communication medium. In case of time-extended tasks, robots can receive information less frequently and the  bandwidth usage can be kept  minimum. However under a fast changing environment relatively more bandwidth will be required.  Finally the centralized communication model spread the attractive fields of all tasks globally by broadcasting information to all robots.  

Our centralized communication model has been implemented by the D-Bus interprocess communication technology \cite{Pennington+2010}. Under this implementation, we have used D-Bus \textit{signal} type asynchronous messages to enable information sharing among SwisTrack multi-robot tracker \cite{Lochmatter+2008}, TPS and RCCs inside a single host. D-Bus signals give us the flexible, fault-tolerant and real-time messaging scheme which can not be easily achieved in other interprocess communication schemes. The detail design and implementation  of our of centralized communication model can be found in \cite{Sarker2010control}. 
%-----------------------------------------------------------
\subsection{Experiments}
We have designed a set of  manufacturing shop-floor scenario experiments for validating the effectiveness of our AFM in producing self-regulated MRTA. In our experiments we design the following observables.
\textbf{Plasticity:} %As we have discussed in Sec. \ref{bg:def:dol},  
Self-regulated MRTA is often characterised by the plasticity and task-specialization, in both macroscopic and microscopic levels. Within our manufacturing shop-floor context, plasticity refers to the collective ability of the robots to switch from doing no-task option (random-walking) to doing a task (or vice-versa) depending on the work-load present in the system. Here we expect to see that most of the robots would be able to engage in tasks when there would be high workloads (or task-urgencies) during PMM. Similarity, when there would be low workload in case of MOM, only a few robots would do the task, rest of them would either be idle (not doing any task) or perform a random-walk.  The changes of task-urgencies and the ratio of robots engaged in tasks can be good metrics to observe plasticity in MRTA.

\textbf{Task-specialization:} Self-regulated MRTA is generally accompanied with task-specializations of agents. That means that few robots will be more active than others. From the interpretation of AFM, we can see that after doing a task a few times, a robot will soon be sensitized to it. Therefore, from the raw log of task-sensitization of robots, we can be able to find the pattern of task-sensitization of robots per task basis.

\textbf{Quality of task-performance:} As discussed in Sec. \ref{afm:vms} we can measure the quality of MRTA from the APCD. It first calculates the ideal minimum production time and then finds the delay in production process from the actual production completion data. Thus this will indicate how much more time is  spent in the production process due to the self-regulation of robots in this distributed task-allocation scheme.

\textbf{Robustness:} In order to see if our system can respond to the gradually increasing workloads,  we can measure APMW within the context of our manufacturing shop-floor scenario. This can show the robustness of our system. When a task is not being served by any robot for some time we can see that its urgency will rise and robots will respond to this dynamic demand. 

\textbf{Flexibility:} From the design of AFM, we know that robots that are not doing a task will be de-sensitized to it or forget that task. So at an overall low work-load (or task urgency), less robots will do the tasks and hence less robots will have the opportunity to learn tasks. From the shop-floor work-load data, we can confirm the presence of flexibility in MRTA.

\textbf{Energy-efficiency:} In order to characterize the energy-efficiency in MRTA we can log the pose data of each robot that can give us the total translations occurred by all robots in our experiments. This can give us a rough indication of energy-usage by our robots. 

\textbf{Information flow:} Since AFM requires a system-wide continuous flow of information, we can measure the communication load to bench-mark our implementation of communication system. This bench-mark data can be used to compare among various communication strategies. Here we can measure  how much task-related information, i.e. task-urgency, location etc. are sent to the robots at each time step. This  amount of information or communication load can be constant or variable depending on the design of the communication system.

\textbf{Scalability:} In order to see the effects of scaling on MRTA, we have designed two group of experiments. Series A corresponds to a small group where we have used 8 robots, 2 tasks under an arena of 2 $m^2$. We have doubled these numbers in Series B, C and D, i.e. 16 robots, 4 tasks under an arena of 4 $m^2$. This proportional design can give us a valuable insight about the effects of scaling on self-regulated MRTA. 

%%
\begin{table}
\caption{Experimental parameters of Series A \& B experiments}
\label{table:params}
\begin{center}
\begin{tabular}{|p{2in}|c|}
\hline Parameter & Series A $\mid$ Series B\\
\hline Total number of robots ($N$) & \hspace*{0.1cm} 8 $\mid$ 16\\
\hline Total number of tasks ($M$) & 2 $\mid$ 4\\
\hline Experiment area ($A$) & 2 $m^2$ $\mid$  4 $m^2$\\
\hline Initial production work load/machine ($\Omega_{j}^{p}$) & 100 unit \\
\hline Task urgency increase rate ($\Delta\phi_{INC}$) & 0.005\\
\hline Task urgency decrease rate ($\Delta\phi_{DEC}$) & 0.0025\\
\hline Initial sensitization ($K_{INIT}$) & 0.1\\
\hline Sensitization increase rate ($\Delta k_{INC}$) & 0.03\\
\hline Sensitization decrease rate ($\Delta k_{DEC}$) & 0.01\\
\hline
\end{tabular}
\end{center}
\end{table}
%%
%------------------------------------------------------------
\subsection{Parameters}
In order to observe the self-regulated MRTA, we have designed our experiments to record the following  observables in each time-step.
\begin{enumerate}
\item Task-urgency of each task ($\phi$).
\item Number of robots engaged in each task.
\item Task-sensitizations ($k$) of robots.
\item Pose data of robots.
\item Communication of task-information message with robots.  
\end{enumerate}
Table \ref{table:params} lists a set of essential parameters of our GSNC strategy based centralized communication experiments (Series A and B). The initial values of task urgencies correspond to 100 units of production work-load without any maintenance work-load as outlined in Eq. \ref{eqn:task-urgency-prod-init}. For task-urgency (and task-sensitization) limits, we choose a limit of 0 and 1, where 0 means no urgency (complete forgetting) of a task and 1 means maximum urgency (or full specialization) of that task. We choose a initial sensitization value of 0.1 for all tasks. 
%
%-------------------------------------------------------------
\subsection{Implementation}
As shown in Fig. \ref{fig:ccm}, in this model there exists a centralized \textit{TaskServer} that is responsible for disseminating task information to robots. The contents of task information can be physical locations of tasks, their urgencies and so on. TaskServer delivers this information by emitting \textit{TaskInfo} signals periodically. For example, in a wireless network it can be a message broadcast. Task-Server has another interface for catching feedback signals from robots. The \textit{RobotStatus} signal can be used to inform TaskServer about a robot's current task id, its device status and so on. TaskServer uses this information to update relevant part of task information such as, task-urgency. This up-to-date information is sent in next TaskInfo signal.
%%
The major components of our implementation are a multi-robot tracking system, robot controller clients and a centralized task-server. In order to track all robots in real-time, we have used SwisTrack \cite{Lochmatter+2008}, a state of the art open-source, multi-agent tracking system, with a 16-megapixel overhead GigE camera. This set-up gives us the position, heading and id of each of the robots by processing the image frames at about 1 FPS. The interaction of the hardware and software of our system is illustrated in Fig. \ref{fig:setup}.
For inter-process communication (IPC), we have used D-Bus technology\footnote{http://dbus.freedesktop.org/doc/dbus-specification.html}. We have developed an IPC component for SwisTrack  that can broadcast id and pose of all robots in real-time over our server's D-Bus interface.

Apart from SwisTrack, we have implemented two major software modules: {\em TaskServer} and {\em Robot Controller Client (RCC)}. They are developed in Python with its state of the art \textit{Multiprocessing}\footnote{http://docs.python.org/library/multiprocessing.html} module. This python module simplifies our need to manage data sharing and synchronization among different sub-processes. As shown in Fig. \ref{fig:setup}, RCC consists of four sub-processes. {\em SignalListener} and {\em SignalEmitter}, interface with SwisTrack D-Bus Server and TaskServer respectively. {\em TaskSelector} implements AFM algorithms for task selection . {\em DeviceController} moves a robot to a target task. Bluetooth radio link is used as a communication medium between a RCC and a corresponding e-puck robot. 
%----------------------------------------------------------------
\subsection{Results and discussions}
Our AFM validation experiments were conducted with 16 robots, 4 tasks in an arena of 4 $m^2$ for about 40 minutes and averaged them over five iterations.
Fig. \ref{fig:raw-urgencies} shows the dynamic changes in task urgencies in one iteration.  In order to describe our system's dynamic behaviour holistically, we analyse the changes in task urgencies over time. Let $ \phi_{j, q}$ be the urgency of a task $j$ at $q^{th}$ time-step and $\phi_{j, q+1}$ be the task urgency of $(q+1)^{th}$ time-step. We can calculate the sum of changes in urgencies of all tasks at $(q+1)^{th}$ time-step:
\begin{equation} 
\small
\Delta \Phi_{j, q+1} = \sum_{j=1}^{M} (\phi_{j, q+1} - \phi_{j, q})
\label{eqn:Delta-Phi}
\end{equation}
From Fig. \ref{fig:urgency-stat} we can see that initially the sum of changes of task urgencies are towards negative direction. This implies that tasks are being served by a high number of robots. Fig. \ref{fig:worker-stat} shows that in production stage, when  work-load is high, many robots are active in tasks and this ratio varies according to task urgency changes.
Fig. \ref{fig:single-robot-sensitizations} gives us the task specialization of five robots on \textit{Task3} in a particular run of our experiment. This shows us how our robots can specialize (learn) and de-specialize (forget) tasks over time. The de-specialization of tasks is calculated similar to Eq. \ref{eqn:Delta-Phi}. We have calculated the absolute sum of changes in sensitizations of all robots by the following equation.
% 
\begin{equation}
\small 
%\Delta K_{j, q+1} = \sum_{j=1}^{M} \left | (k_{j, q+1} - k_{j,q}) \right |
\Delta K_{j, q+1} = \sum_{j=1}^{M}  |(k_{j, q+1} - k_{j,q})| 
\label{eqn:Delta-K}
\end{equation}
This values of $\Delta K$ are plotted in Fig. \ref{fig:sensitization-stat}. It shows that the overall rate of learning decreases and forgetting increases over time. It is a consequence of the gradually increased task specialization of robots and reduced task-urgencies over time.
%%
%%% Communication load %%%
Fig. \ref{fig:signal-frequency-stat} presents the frequency of signalling task information by TaskServer. Since the duration of each time step is 50s long and TaskServer emits signal in every 2.5s, there is an average of 20 signals in each time-step.

Within our manufacturing shop-floor scenario, we have got average production completion time 165 time-steps (825s) where sample size is (5 x 4) =  20 tasks, SD = 72 time-steps (360s). According to Eq. \ref{eqn:min-pmm}, our theoretical minimum  production completion time is 50 time-steps (250s) assuming the non-stop task performance of all 16 robots with an initial task urgency of 0.5 for all 4 tasks and  task urgency decrease rate $\Delta \Phi_{DEC	}$ = 0.0025 per robot per time-step. Hence, Eq. \ref{eqn:appd} gives us APCD, $\zeta$ = 2.3 which means that our system has taken 2.3 times more time (575s) than the estimated minimum time.
Besides,  from the average 315 time-steps (1575s) maintenance activity of our robots per experiment run, we have got  APMW, $\chi$ = 0.012756  which corresponds to the pending work of 3 time-steps (15s) with sample-size = 20 tasks, SD = 13 time-steps (65s), where $\Delta \Phi_{INC}$ = 0.005 per task per time-step. This tells us the robust task performance of our robots which can return to an abandoned task within a minute or so.
%%
\begin{figure}
\begin{minipage}[t]{0.48\linewidth}
\centering
\includegraphics[width=6cm, height=4cm, angle=0]
{images/PlotUrgencyLog-2010May10-115549.eps}
%figure caption is below the figure
\caption{\small Dynamic task-urgency changes.}
\label{fig:raw-urgencies} % Give a unique label
\end{minipage}
\hspace{0.5cm}
\begin{minipage}[t]{0.48\linewidth}
\centering
\includegraphics[width=6cm, height=4cm, angle=0]{images/TaskUrgencyStat.eps}
\caption{\small Shop-floor workload history} % measured in terms of task urgencies
\label{fig:urgency-stat} % Give a unique label
\end{minipage}
\end{figure}
%%
%%% Sensitization and Translation %%%
\begin{figure}
\begin{minipage}[t]{0.48\linewidth}
\centering
%\includegraphics[height=4.8cm, angle=0]
\includegraphics[width=6cm, height=4cm, angle=0]
{images/Global-SignalingFreqStat.eps}
%figure caption is below the figure
\caption{\small Task server's task-info broadcasts}
\label{fig:signal-frequency-stat}
%
\end{minipage}
\hspace{0.5cm}
\begin{minipage}[t]{0.48\linewidth}
\centering
%\includegraphics[height=4.8cm, angle=0]
\includegraphics[width=6cm, height=4cm, angle=0]
{images/SB-WorkerRatio.eps}
\caption{\small Self-organized task-allocation}
\label{fig:worker-stat} % Give a unique label
\end{minipage}
\end{figure}
%%
%\begin{figure}
%\begin{minipage}[t]{0.48\linewidth}
%\centering
%\includegraphics[height=4cm, angle=0]{images/TaskSpecialization-task3-10may-1.eps}
%\caption{\small Task specialization on Task3}
%\label{fig:single-robot-sensitizations} % Give a unique label
%\end{minipage} 
%%%%


%======================================================================
\section{Comparisons between local and centralized communication strategies}
In most swarm robotic research local communication is considered as the one of the most critical components of the swarms where the global behaviours emerges from the local interactions among the individuals and their environment. In this study, we have used the concepts of pheromone active-space of ants to realize our simple LSLC scheme. Ants use various chemical pheromones with different active spaces (or communication ranges) to communicate different messages with their group members \cite{Holldobler1990}. Ants sitting near the source of this pheromone sense and respond quicker than others who wander in far distances. Thus both communication and sensing occurs within a small communication range\footnote{Although, generally communication and sensing are two different issues, however within the context of our self-regulated MRTA, we have broadly viewed sensing as the part of communication process, either implicitly via environment, or explicitly via local peers.}. We have used this concept of communication range or locality in our LPCM. A suitable  range (or radius) of communication and sensing can be set at design time based on the capabilities of robots \cite{Agassounon+2002}. Alternately they can also be varied dynamically over time depending on the  cost of communication and sensing, e.g. density of peers, ambient noise in the communication channels, or even by aiming for maximizing information spread  \cite{Yoshida+2000}. In this study, we have followed the former approach as our robots do not have the precise hardware to dynamically vary their communication and sensing ranges. Below we describe the general characteristics and implementation algorithm and strategy of Local P2P communication model (LPCM).
%%---------------------------------------------------------
\subsection{General characteristics of LPCM}
Our LPCM relies on the local P2P communications among robots. we have assumed that robots can communicate to its nearby peers within a certain communication radius, $r_{comm}$ and they can sense tasks within another radius $r_{task}$. They exchange communication signals reliably without any significant loss of information. A robot $R_1$ is a {\em peer} of robot $R_2$, if spatial distance between $R_1$ and $R_2$ is less than this $r_{comm}$.
Similarly, when a robot comes within this $r_{task}$ of a task, it can sense the status of this task. Although the communication and sensing  range can be different based on robot capabilities, we have considered them same for the sake of simplicity of our implementation.

Local communication can also give robots similar task information as in centralized communication. In this case, it is not necessary for each robot to communicate with every other robot to get information on all tasks. Since robots can random walk and explore the environment we assume that for a reasonably high robot-to-space density, all tasks will be known to all robots after an initial exploration period. In order to update the urgency of a task, robots can estimate the number of robots working on a task in two ways:  by either using their sensory perception (e.g. on-board camera) or  doing local P2P communication with others.

Similar to our centralized communication model, we can characterize our local communication model in terms of message content, communication frequency and target recipients \cite{Gerkey+2001}. Regarding the issue of message content, our local communication model is open. Robots can communicate with their peers with any kind of message. Our local model addresses the last two issues very specifically. Robots communicate only when they meet their peers within a certain communication radius ($r_{comm}$). Although in case of an environment where robots move relatively faster the peer relationships can also be changed dynamically. But this can be manipulated by setting the signal frequency and robot to space density to somewhat reasonably higher value.

In terms of target recipients, our model differs from a traditional publish/subscribe communication model by introducing the concept of dynamic subscription. In a traditional publish/subscribe communication model, subscription of messages happens prior to the actual message transmission. In that case prior knowledge about the subjects of a system is necessary. But in our model this is not necessary as long as all robots uses a common addressing convention for naming their incoming signal channels. In this way, when a robot meets with another robot it can infer the address of this peer robot's channel name by using a shared rule. A robot is thus always listening to its own channel for receiving messages from its potential peers or message publishers. On the other side, upon recognizing a peer, a robot sends a message to this particular peer. So here neither it is necessary to create any custom subject name-space  \cite{Gerkey+2001} nor we need to hard-code information in each robot controller about the knowledge of their potential peers {\em a priori}. 

In order to implement LPCM, our centralized communication scheme has been converted into a decentralized one where robots can use local observation and communicate with peers about tasks to estimate task-urgencies. Under this implementation, we present an emulation of this scenario where robots do not depend entirely on TPS for estimating task-urgencies, instead they get task information from TPS when they are very close to a task (inside $r_{task}$) or from local peers who know about a task via TPS. %The specific implementation of P2P signals are discussed in 
%
The sample raw task-urgencies of Series C and Series D experiments are shown in Fig. \ref{fig:raw-urgencies-SC-SD}. In case of Series D, we can see that an unattended task, \textit{Task4}, was not served by any robot for a long period and later it was picked up by some of the robots. 
%%% raw urgencies
\begin{figure}
\centering
\hspace*{0.5cm}
\subfigure[Task-urgency]
{
\includegraphics[width=6cm, height=4cm]{images/SC-PlotUrgencyLog-2010Feb15-171017.eps}
\hspace{0.25cm}
%\ [Series D]
\includegraphics[width=6cm, height=4cm]{images/SD-PlotUrgencyLog-2010Feb17-112141.eps}
}
\caption{\small Task-urgencies observed in (a) Series C and (b) Series D experiments}
\label{fig:raw-urgencies-SC-SD} 
\end{figure}
%%
%% – Workload
\begin{figure}
\centering
\hspace*{0.5cm}
\subfigure[Shop-floor workload]
{
\includegraphics[width=6cm, height=4cm]{images/SC-TaskUrgencyStat.eps}
\hspace{0.25cm}
%\ [Series D]
\includegraphics[width=6cm, height=4cm]{images/SD-TaskUrgencyStat.eps}
}
\caption{\small Shop-floor work-load history (a) Series C and (b) Series D experiments}
\label{fig:workload-SC-SD} 
\end{figure}
%%%
The dynamic shop floor work-load is shown in Fig. \ref{fig:workload-SC-SD}. These plots shows similar work-load as experienced in Series A and Series B experiments. Here, we can also see that initially the sum of changes of task urgencies are towards negative direction. This implies that tasks are being served by a high number of robots. When the task urgencies stabilize near zero the fluctuations in urgencies become minimum. Since robots chose tasks stochastically, there will always be a small changes in task urgencies.
%
%%----------------------------------------------------------------
\subsection*{Ratio of active workers}
% Plasticity
\begin{figure}
\centering
\hspace*{0.5cm}
\subfigure[Self-organized worker allocation]
{
\includegraphics[width=6cm, height=4cm]{images/SC-Local50cm-Plasticity.eps}
\hspace{0.25cm}
%\ [Series D]
\includegraphics[width=6cm, height=4cm]{images/SD-Local1m-Plasticity.eps}
}
\caption{\small Self-organized allocation of workers in (a) Series C and (b) Series D}
\label{fig:plasticity-SC-SD} 
\end{figure}
%
The active worker ratios of both Series C and Series D experiments are plotted in Fig. \ref{fig:plasticity-SC-SD}. Series C data shows us a large variation in this active worker ratios.
%%
%%-------------------------------------------------------------
%\subsection*{Shop-task performance}
%The task-performance of our manufacturing shop-floor scenario under both Series C and Series D experiments are plotted in Fig. \ref{fig:vms-SC-SD}. 
%\begin{figure}
%\centering
%\hspace*{0.5cm}
%\ [APCD]
%{\includegraphics[height=8cm]{images/apcd.eps}}
%\newline
%\ [APMW]
%{\includegraphics[height=8cm]{images/apmw.eps}}
%\caption{\small Task-performance of our shop-floor manufacturing scenario (a) APCD and (b) APMW}
%\label{fig:vms-SC-SD} 
%\end{figure}
From task-performance data of Series C we have got average production completion time 121 time-steps (605s) with SD = 36 time-steps (180s). For Series D,  average production completion time is 123 time-steps (615s) with SD = 40 time-steps (200s). According to Eq. \ref{eqn:min-pmm}, our theoretical minimum production completion time is 50 time-steps (250s) as discussed in Sec \ref{afm:results}.  The values of APCD are as follows. For Series C, $\zeta$ = 1.42 and for Series D, $\zeta$ = 1.46. For both series of experiments APCD values are very close.\\
%%
For APMW, Series C experiments give us an average time length of 359 time-steps (1795s). In this period we calculated APMW and it is 5 time-steps with SD = 17 time-steps and $\chi$ = 0.023420. For Series D experiments, from the average 357 time-steps (1575s) of maintenance activity of our robots per experiment run, we have got APMW, $\chi$ = 0.005359 which corresponds to the pending work of 2 time-steps (10s) where SD = 7 time-steps.
%%-------------------------------------------------

%%
\begin{figure}
\begin{minipage}[t]{0.48\linewidth}
\centering
\includegraphics[width=6cm, height=4cm, angle=0]{images/K-Group.eps}
\caption{ Overall task-specialization of robot groups.}
\label{fig:K-Group} 
\end{minipage} 
\hspace{0.5cm}
\begin{minipage}[t]{0.48\linewidth}
\centering
\includegraphics[width=6cm, height=4cm, angle=0]
{images/Q-Group.eps}
\caption{Time-steps to reach the peak values of task-specialization}
\label{fig:Q-G-SC-SD} 
\end{minipage}
\end{figure}
%%
By applying Eq. \ref{eqn:K-G} and Eq. \ref{eqn:Q-G} on our robots' task-sensitization statistics, we have got the peak task-sensitization $K^G_{avg}$ values: 0.39 (SD=0.17) and 0.27 (SD=0.10), and their respective time-step $Q^G_{avg}$ values: 13 (SD=7) and 11 (SD=5) time-step. They are shown in Fig. \ref{fig:K-G-SC-SD} and Fig. \ref{fig:Q-G-SC-SD}. Here we can see that the robots in Series C exhibited higher task-specialization than that of Series D experiments.\\
%%-------------------------------------------------
%\subsection*{Robot motions}
%%%%% Traslation local 
\begin{figure}
\centering
\hspace*{0.5cm}
\subfigure[Robot motion statistics]
{
\includegraphics[width=6cm, height=4cm]
{images/SC-DeltaTranslationStat.eps}
%figure caption is below the figure
\hspace{0.25cm}
%\centering
%\ [Series D]
%{
\includegraphics[width=6cm, height=4cm]{images/SD-DeltaTranslationStat.eps}
}
\caption{\small Sum of translations of all robots (a) Series C and Series D experiments }
\label{fig:translation-SC-SD} 
\end{figure}
%%
We have aggregated the changes in translation motion of all robots over time following Eq. \ref{eqn:Delta-Tr}. The robot translation results from both local mode experiments are plotted in Fig. \ref{fig:translation-SC-SD}. In this plot we can see that robot translations also vary over varying task requirements of tasks.\\ 
%%-------------------------------------------------
%\textbf{Communication load}
%%% Communication load %%%
\begin{figure}
\centering
\hspace*{0.5cm}
\subfigure[Communication load]
{
\includegraphics[width=6cm, height=4cm]
{images/SC-Local-500cm-SignalingFreqStat.eps}
\hspace{0.25cm}
%\ [Series D]
\includegraphics[width=6cm, height=4cm]{images/SD-Local-1m-SignalingFreqStat.eps}
}
\caption{\small Frequency of overall LocaltaskInfo (P2P) signalling in (a) Series C and (b) Series D experiments}
\label{fig:local-signal-frequency-stat} % Give a unique label
\end{figure}
%
%-----------------------------------------------------------
\subsection{Comparisons}
Results from Series C and Series D experiments show us many similarities and differences with respect to the results of Series A and Series B experiments. Both Series C and Series D experiments show similar APCD values: 1.42 and 1.46 respectively, which are significantly less than Series B experiment result (APCD = 2.3) and are close to Series A experiment result (APCD = 1.22). This means that for large group, task-performance  is efficient under LSLC strategy (Series C and Series D) comparing with their GSNC counterpart (Series B).

Besides, in terms of task-specialization, the overall task-specialization of group in Series C ($K^G_{avg}$ = 0.4) is  closer to that of Series A experiments ($K^G_{avg}$ = 0.39) and interestingly, the value of  Series D ($K^G_{avg}$ = 0.27) is  much closer to that of Series B experiments ($K^G_{avg}$ = 0.30). So task-specialization in large group under LSLC strategy shows higher performance than their GSNC counter part. Besides task-specialization happens much faster under LSLC strategy as we can see that the average time to reach peak sensitization values  of the group,  $Q^G_{avg}$ in Series C is lower than that of Series A values by 25 time-steps.

\begin{table}
\begin{center}
\caption{Sum of translations of robots in Series A-D experiments.}
\begin{tabular}{|c|c|c|}
\hline \textbf{Series} & \textbf{Average translation (m)} & \textbf{SD} \\ 
\hline A & 2.631 & 0.804\\ 
\hline B & 13.882 & 3.099\\
\hline C & 4.907 & 1.678\\
\hline D & 4.854 & 1.592\\
\hline
\end{tabular}
\label{table:motion-cmp} 
\end{center}
\end{table}
%%
From the robot motion profiles found in all four series of experiments, we have found that under LSLC strategy, robot translations have been reduced significantly. Table \ref{table:motion-cmp} summarizes the average translations done by robots in all four series of experiments. From this table we can see than Series C and Series D show about 2.8 times less translation than that in Series B experiments. The translation of 16 robots in Series C and Series D experiments are approximately double (1.89 times) than that of Series A experiments with 8 robots.  Thus the energy-efficiency under LSLC strategy seems to be higher  than that under GSNC strategy.

From the above results we can see that large group robots achieve better MRTA under LSLC strategy. The local sensing of tasks prevents them to attend a far-reaching task which may be more common under global sensing strategy. However, as we have seen in Fig. \ref{fig:raw-urgencies-SC-SD}
some tasks can be left unattended for a long period of time due to the failure to discover it by any robot. For that reason we see that the values of APMW is slightly higher under LSLC strategy. But this trade-off is worth as LSLC strategy provides superior self-regulated MRTA in terms of task-performance, task-specialization and energy-efficiency.
%=======================================================================
\section{Conclusion}
This study has focused on reviewing bio-inspired communication strategies for self-regulated multi-robot systems with an emphasis on comparing two bio-inspired  communication and sensing strategies in producing self-regulated MRTA by an interdisciplinary model of DOL, AFM. Under the GSNC strategy, AFM has produced the desired self-regulated MRTA among a group of 8 and 16 robots. This gives us the evidence that AFM can successfully solve the MRTA issue of a complex multi-tasking environment like a manufacturing shop-floor. Under the LSLC strategy, AFM can also produce the desired self-regulated MRTA for 16 robots with different communication and sensing ranges.

From our comparative results, we can conclude that for large group of robots,  degradation in  task-performance and task-specialization of robots are likely to occur  under GSNC strategy that relies upon a centralized communication system. Thus GSNC strategy can give us better performance when the number of tasks and robots are relatively small. This confirms us the assertions made by some biologists that self-regulated DOL among small group of individuals can happen without any significant amount of local communications and interactions. However, our findings suggest that task-specialization can still be beneficial among the individuals of a small group which contradicts the claim that small groups only posses the generalist workers, but not the specialists.

On the other hand, LSLC strategy is more suitable for large group of individuals that are likely to be unable to perform global sensing and global communications with all individuals of the group. The design of communication and sensing range is still remained as a critical research issue. However, our results suggest that the idea of maximizing information gain is not appropriate under a stochastic task-allocation process, as more information causes more task-switching behaviours that lowers the level of task-specialization of the group. This might not be the case under a deterministic task-allocation scheme where more information leads to better and optimum allocations, but that is limited to a small group of individuals. Nevertheless, despite having the limited communication and sensing range, LSLC strategy helps to produce comparatively better task-allocation with increased task-specialization and significantly reduced motions or savings in energy e.g. battery power.

Our study has experienced all the major challenges of implementing a large multi-robot system within limited time and resource constraints. From our experiences we can say that, by following flexible and open-source hardware and software platforms, it is possible to construct a large multi-robot system for conducting real-robotic experiments within limited time-frame. This is a good news for those who wants to test their models using a large real-robotic system without relying on simulation only. However in this case, one needs to adopt the an appropriate test-driven bottom-up development approach, instead of following the traditional top-down ``model - simulate -  export-to-real-robots'' approach. 
%=======================================================================
\section{Future works}
In this study, we have kept the interaction among robots for task-completion under manufacturing shop-floor scenario as simple as possible. This is done mainly for minimizing the time and complexity of real-robotic implementation. However, in most of the instances of biological self-regulated DOL, e.g. among polybia wasps that follow LSLC strategies for DOL, several inter-dependent tasks are often performed concurrently with a high degree of interaction among individuals. So this study can possibly be extended in co-operative task performance where different individuals with variety of task-skills need to interact with each other directly. For example, we can consider complex manufacturing shop-floor scenarios where the assembly of a machine part requires coordination among multiple tasks. 

Our validation of AFM has been limited to a group of homogeneous robots that has initially same level of task-sensitization. Moreover no dynamic task has been introduced during the run-time of our experiments. Due to the stochastic task-allocation process, we always were able to see the  variation in task-urgencies. But AFM can be applied to a more challenging environment with suddenly appearing (and disappearing) dynamic tasks that can resemble to the real-world use-cases where any task can be pre-empted by other tasks. Moreover, some more research can be done in order to figure out how to optimize the initial experimental parameters, e.g. robots' task learning and forgetting rate etc. 

In terms of implementation of our LSLC strategy, P2P communication among robot-controllers occurred in host-PC. We did not validate the use of communication range  using real-robotic hardware. But real implementation of communication range can easily be achieved by using suitable on-board communication module, e.g. Wifi or IR, with  relatively powerful robots.

In our experiments, we have selected two fixed communication ranges with an approximation of LSLC strategy. Some researchers have addressed the issue of deciding the optimum communication range \cite{Yoshida+2000} without any direct connection with self-regulated MRTA. However, much research is required to find optimum communication range as a property of self-regulation of individuals. For example, in Sec. \ref{bg:bio-comm:comm-role} we have seen that urgency of a task directly influences the communication behaviours of individuals, e.g. honey-bees modulate their dancing behaviours based on the quality of flower source.

Finally, in terms of real-world implementation, we can consider to put AFM for solving many challenging industrial automation tasks. For example, AFM can solve automated material handing tasks in ware-houses and factories, or real manufacturing tasks with suitable hardware module. In this way, our interdisciplinary model can help to overcome the existing challenges in the industry.


%%% With a bib file, include it! *************************************
\bibliography{intech-book}
 
\end{document}
