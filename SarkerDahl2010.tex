%\documentclass[draft]{intech}
\documentclass{intech}
%\usepackage{your_package}	%if you need custom package
%\usepackage[notquote]{hanging}
\usepackage[nolist]{acronym}
%\usepackage[colorlinks=false]{hyperref}
%\usepackage{graphicx,hypernat,subfig}
%\usepackage{color}
\usepackage[nolist]{acronym}
\usepackage{multirow,multicol}
\usepackage{booktabs}
\usepackage{rotating}
\usepackage{listings,program,threeparttable}
\usepackage{algorithmicx,algpseudocode}

% * CHAPTER NUMBER * BOOK NAME * AUTHOR(S) NAME *****************************
\setcounter{chapter}{0} % It will be set by technical editor.

\booktitle{Will-be-set-by-IN-TECH}%

\chaptertitle{Bio-inspired Communication for Self-regulated
Multi-robot Systems} % You know your chapter title?

\authors{Md Omar Faruque Sarker and Torbj{\o}rn S. Dahl}
\affiliation{University of Wales, Newport}
\country{United Kingdom}


% END * CHAPTER NUMBER * BOOK NAME * AUTHOR(S) NAME *************************

\begin{document}

\maketitle

%%
\begin{acronym}
%==== A B C D ====
\acro{AFM}{attractive field model}
\acro{AGV}{automated guided vehicle}
\acro{APCD}{average production completion delay}
\acro{APMW}{average pending maintenance work-load}
%\acro{AS}{active space}
\acro{BMS}{biology-inspired manufacturing system}
\acro{CCD}{charge-coupled device}
%\acro{CCM}{centralized communication mode}
\acro{DEM}{data and event management}
\acro{DOL}{division of labour}
%==== E F G H ====
%\acro{EPSRC}{Engineering and Physical Sciences Research Council}
\acro{GigE}{Gigabyte Ethernet}
\acro{GIL}{Global Interpreter Lock}
%\acro{GPS}{global positioning system}	
\acro{GSNC}{global sensing - no communication}
%\acro{GUI}{graphical user interface}
\acro{HEAD}{hybrid event-driven architecture on D-Bus}
%====  I J K L ====
%\acro{IF}{independent founders}
%\acro{INS}{indoor navigation system}
\acro{IPC}{inter-process communication}
%\acro{IR}{infrared}
%\acro{LCM}{local communication mode}
\acro{LSLC}{local sensing - local communication}
%==== M N O P ====
\acro{MOM}{maintenance only mode}
\acro{MRS}{multi-robot system}
\acro{MRTA}{multi-robot task allocation}
\acro{P2P}{peer-to-peer}
%\acro{PF}{potential field}
\acro{PMM}{production and maintenance mode}
%=====  Q R S T ====
\acro{RCC}{robot-controller client}
%\acro{RW}{random walk}
\acro{SDK}{software-development kit}
%\acro{SF}{swarm founders}
\acro{SHM}{shared memory}
%\acro{SI}{swam intelligence}
%\acro{SO}{self-organization}
%\acro{SR}{self-regulation}
%\acro{SRS}{swarm robotic system}
\acro{TPS}{task perception server}
%==== U V W X ====
%==== Y Z ==== 
\end{acronym}
%%
\section{Introduction}
In recent years, the study of biological social insects and other animals reveals us that simple individuals of these self-organized societies can solve various complex and large everyday-problems with a few behavioural rules \citep{Camazine+2001}. In these self-organized systems, an individual agent may have limited cognitive, sensing and communication capabilities. But they are collectively capable of solving complex and large problems, e.g. coordinated nest construction of honey-bees, collective defence of school fishes from a predator attack. Since the discovery of these collective behavioural patterns of self-organized societies, scientists observed modulation of behaviours in the individual level \citep{Garnier+2007}. One of the most notable such self-regulatory processes in biological social systems is the \textit{division of labour} (DOL) \citep{Sendova-Franks+1999} by which a larger task is divided into a number of small subtasks and each subtask is performed by a separate individual or a group of individuals. {\em Task-specialization} is an integral part of DOL where a worker usually does not perform all tasks, but rather specializes in a set of tasks, according to its morphology, age, or chance \citep{Bonabeau+1999}. DOL is also characterized by {\em plasticity} which means that the removal of one class of workers is quickly compensated for by other workers. Thus distributions of workers among different concurrent tasks keep changing according to the environmental and internal conditions of a colony. 

In artificial social systems, like multi-agent or multi-robot system, the term ``division of labour'' is often found synonymous to ``task-allocation'' \citep{Shen+2001}. In robotics, this is called \textit{multi-robot task allocation} (MRTA) which is generally identified as the question of assigning tasks in an appropriate time to the appropriate robots considering the changes of in task-requirements, environment and the performance of other team members. The complexities of the distributed MRTA problem arise from the fact that there is no central planner or coordinator for task assignments, and generally robots have limited capabilities to sense, to communicate and to interact locally. In this chapter, we have presented this issue of DOL as the representative self-regulatory process in both biological and artificial social systems. We have used the terms DOL and MRTA (or simply, task-allocation) interchangeably.
 
Traditionally, task allocation in a multi-agent systems has been dominated by explicit and self-organized task-allocation approaches. Explicit approaches, e.g. intentional cooperation \citep{Parker2008}, use of dynamic role assignment \citep{Chaimowicz2002} and market-based bidding approach \citep{Dias+2006} are intuitive, comparatively straight forward to design and implement and can be analysed formally. However, these approaches typically works well only when the number of robots are small ($\leq 10$) \citep{Lerman+2006}. On the other hand bio-inspired self-organized task-allocation relies on the emergent group behaviours, such as emergent cooperation \citep{Kube+1993}, adaptation rules \citep{Liu+2007} etc. They are more robust and scalable to large team sizes. However, this  approach is difficult to design, to analyse formally and to implement in real robots. Existing researches under this approach mostly limit their focus on one specific global task \citep{Gerkey+2004}.

Within the context of the Engineering and Physical Sciences Research Council (EPSRC) project, ``Defying the Rules: How Self-regulatory Systems Work'', we have proposed to solve the above mentioned self-regulated DOL problem in an alternate way \citep{Arcaute+2008}. Our approach is inspired from the studies of emergence of task-allocation in both biological and human social systems. We have proposed four generic requirements to explain self-regulation in those social systems. These four rules are: \textit{continuous flow of information}, \textit{concurrency}, \textit{learning} and \textit{forgetting}. Primarily these rules deal with the issue of deriving local control laws for regulating an individual's task-allocation behaviour that can facilitate the DOL in the entire group. In order to employ these rules in the individual level, we have developed a formal model of self-regulated DOL, called the \textit{attractive field model} (AFM). Section 2 reviews our generic requirements of self-organization and AFM.

In biological social systems, communication among the group members and  sensing the task-in-progress, are two key components of self-organized DOL. In robotics, existing self-organized task-allocation methods rely heavily upon local sensing and local communication of individuals for achieving self-organized task-allocation. However, AFM differs significantly in this point by avoiding the strong dependence on the local communications and interactions. AFM requires a system-wide continuous flow of information about tasks, agent states etc. but this can be achieved by using both centralized and decentralized communication modes under explicit and implicit communication strategies. 

In order to enable continuous flow of information in our multi-robot system, we have implemented two types of sensing and communication strategies inspired by the self-regulated DOL found in two types of social wasps: {\em polistes} and {\em polybia} \citep{Jeanne1999}. Depending on the group size, these species follow different strategies for communication and sensing of tasks. Polistes wasps are called the {\em independent founders} in which reproductive females establish colonies alone or in small groups (in the order of $10^2$), but independent of any sterile workers. On the other hand, polybia wasps are called the {\em swarm founders} where a swarm of workers and queens initiate colonies consisting of several hundreds to millions of individuals. The most notable difference in the organization of work of these two social wasps is: independent founders do not rely on any cooperative task performance while swarm founders interact with each-other locally to accomplish their tasks. The work mode of independent founders can be considered as {\em global sensing - no communication (GSNC)} where the individuals sense the task requirements throughout a small colony and do these tasks without communicating with each other. On the other hand, the work mode of swarm founders can be treated as {\em local sensing - local communication (LSLC)} where the individuals can only sense tasks locally due to large colony-size and they can communicate locally to exchange information, e.g. task-requirements (although their exact mechanism is unknown). In this chapter, we have used these two sensing and communication strategies to compare the performance of the self-regulated DOL of our robots under AFM.
%===================================================================
\section{The Attractive Field Model}
\label{afm:framework}
Inspired from the DOL in ants, humans and robots, we have proposed the following necessary and sufficient set of four requirements for self-regulation in social systems.

\textbf{Requirement 1: Concurrence.} The simultaneous presence of several task options is necessary in order to meaningfully say that the system has organised into a recognisable structure.   In task-allocation terms the minimum requirement is a single task as well as the option of not performing any task.

\textbf{Requirement 2: Continuous flow of information.} Self-organised social systems establish a flow of information over the period of time when self-organisation can be defined.  The task information provides the basis on which the agents self-organise by enabling them to perceive tasks and receive feedback on system performance.

\textbf{Requirement 3: Sensitization.} The system must have a way of representing the structure produced by self-organisation, in terms of MRTA, which tasks the robots are allocated.  One of the simplest ways of representing this information is an individual preference parameter for each task-robot combination.  A system where each robot has different levels of preference or {\em sensitivity} to the available tasks, can be said to have to embody a distinct organisation through differentiation.

\textbf{Requirement 4: Forgetting.} When a system self-organises by repeated increases in individual sensitisation levels, it is also necessary, in order to avoid saturation, to have a mechanism by which the sensitisation levels are reduced or {\em forgotten}.  Forgetting also allows flexibility in the system, in that the structure can change as certain tasks become important and other tasks become less so. % This effect can be achieved by mechanisms such as a slow general decay of sensitisation values or explicit negative feedback.
%%
\begin{figure}
\centering
\includegraphics[height=5cm, angle=0]{./images/AFM-Diag3.eps}
\caption{The attractive filed model (AFM)}
\label{fig:afm} % Give a unique label
\end{figure}

Building on the requirements for self-organised social systems, AFM formalises these requirements in terms of the relationships between properties of individual agents and of the system as a whole \citet{Arcaute+2008}.  AFM is a bipartite network, i.e. there are two different types of nodes.  One set of nodes describes the sources of the attractive fields, the tasks, and the other set describes the agents.  Edges only exist between different types of nodes and they encode the strength of the attractive field as perceived by the agent.  There are no edges between agent nodes.  All communication is considered part of the attractive fields.  There is also a permanent field representing the {\em no-task} option of not working in any of the available tasks.  This option is modelled as a random walk.  The model is presented graphically in Fig. \ref{fig:afm}.  The elements are depicted as follows.
%\begin{enumerate}
%\item 
Source nodes (o) are tasks to be allocated to agents.
%\item 
Agent nodes (x) e.g., ants, humans, or robots.
%\item 
Black solid edges represent the attractive fields and correspond to an agent's perceived stimuli from each task.
%\item 
Green edges represent the attractive field of the ever present no-task option, represented as a particular task (w).
%\item 
The red lines are not edges, but represent how each agent is allocated to a single task at any point in time.
%\end{enumerate}
The edges of the AFM network are weighted and the value of this weight describes the strength of the stimulus as perceived by the agent.  In a spatial representation of the model, the strength of the field depends on the physical distance of the agent to the source.  In information-based models, the distance can represent an agent's level of understanding of that task.  The strength of a field is increased through the sensitisation of the agent through experience with performing the task.  This elements is not depicted explicitly in Figure~\ref{fig:afm} but is represented in the weights of the edges.  
In summary, from the above diagram of the network, we can see that each of the agents is connected to each of the tasks. This means that even if an agent is currently involved in a task, the probability that it stops doing it in order to pursue a different task, or to random walk, is always non-zero.

AFM assumed a repeated task selection by individual agents.  The probability of an agent choosing to perform a task is proportional to the strength of the task's attractive field, as given by Equation~\ref{eqn:afm3}.
\begin{equation}
P_{j}^{i} = \frac{S_{j}^{i}}{\sum_{j=0}^{J} S_{j}^{i}} \hspace*{0.25cm}where,\hspace*{0.25cm}S^{i}_{0} = S^{i}_{RW}   
\label{eqn:afm3}
\end{equation}
Equation~\ref{eqn:afm3} states that the probability of an agent, $i$, selecting a task, $j$, is proportional to the stimulus, $ S^i_j$, perceived from that task, with the sum of all the task stimuli normalised to $1$.

The strength of an attractive field varies according to how sensitive the agent is to that task, $k_{j}^{i}$, the distance between the task and the agent, $d_{ij}$, and the {\em urgency}, $\phi _{j}$ of the task.  In order to give a clear edge to each field, its value is modulated by the hyperbolic tangent function, $tanh$.  Equation~\ref{eqn:afm1} formalises this part of AFM.
%% S
\begin{equation}
S_{j}^{i} = tanh\{\frac{k_{j}^{i}}{d_{ij}+\delta } \phi _{j}\}
\label{eqn:afm1}
\end{equation}
Eqation~\ref{eqn:afm1}, used small constant $\delta$, called {\em delta distance}, to avoid division by zero, in the case when a robot has reached to a task.

Equation~\ref{eqn:afm2} shows how AFM handles the the no-task, or random walk, option.  The strength of the stimuli of the random walk task depends on the strengths of the fields real tasks.  In particular, when the other tasks have a low overall level of sensitisation, i.e., relatively weak fields, the strength of the random walk field if relatively high.  On the other hand, when the agent is highly sensitised, the strength of the random walk field becomes relatively low.  We use $J$ to denote the number of real tasks.  AFM effectively considers random walking as an ever present additional task.  Thus the total number of tasks becomes $J+1$. %--P(Task)
\begin{equation}
S^{i}_{RW} = tanh \left \{ 1 -  \frac{ \sum_{j=1}^{J} S^{i}_{j}}{J + 1} \right \}
\label{eqn:afm2}
\end{equation}
%-- P(RW)

A task $j$ has an associated urgency $\phi_j$ indicating its relative importance over time.  If an agent attends a task $j$ in time step $t$, the value of $\phi_j$ will decrease by an amount $\delta_{\phi_{INC}}$ in the time-step $t+1$.  On the other hand, if a task has not been served by any of the agents in time-step $t$, $\phi_j$ will increase by a different amount, $\delta_{\phi_{DEC}}$ in time-step $t+1$.  This behaviour is formalised in Equations~\ref{eqn:delta-phi1} and~\ref{eqn:delta-phi2}.
\begin{equation}
 If\hspace*{0.15cm}the\hspace*{0.15cm} task\hspace*{0.15cm}is\hspace*{0.15cm}not\hspace*{0.15cm}being\hspace*{0.15cm} done:\hspace*{0.15cm} \phi_{j,t+1} \rightarrow \phi_{j,t} \hspace*{0.15cm} + \delta_{\phi_{INC}}
\label{eqn:delta-phi1}
\end{equation}
%%
\begin{equation}
 If\hspace*{0.15cm}the \hspace*{0.15cm}task\hspace*{0.15cm}is\hspace*{0.15cm}being\hspace*{0.15cm}done:\hspace*{0.15cm}  \phi_{j,t+1} \rightarrow \phi_{j,t} \hspace*{0.15cm} - n\hspace*{0.10cm}\delta_{\phi_{DEC}}
\label{eqn:delta-phi2}
\end{equation}
Equation~\ref{eqn:delta-phi1} refers to a case where no agent attends to task $j$ and Equation~\ref{eqn:delta-phi2} to the case where $n$ agents are concurrently performing task $j$.

In order to complete a task, an agent needs to be within a fixed distance of that task.  When an agent performs a task, it learns about it and this will increases the probability of that agent selecting that task in the future.  This is done by increasing its sensitization to the task by a fixed amount, $k_{INC}$. The variable affinity of an agent, $i$, to a task, $j$, is called its {\em sensitization} to that task and is denoted $k^{i}_{j}$.  If an agent, $i$, does not do a task $j$, $k^i_j$ is decreased by a different fixed amount, $k_{DEC}$.  This behaviour is formalised in Equations~\ref{eqn:k-inc} and~\ref{eqn:k-dec}.
\begin{equation}
 If\hspace*{0.15cm}task\hspace*{0.15cm}is\hspace*{0.15cm}done:\hspace*{0.15cm}  k^i_j \rightarrow   k^i_j \hspace*{0.15cm} + \hspace*{0.15cm} k_{INC}
\label{eqn:k-inc}
\end{equation}
\begin{equation}
 If\hspace*{0.15cm}task\hspace*{0.15cm}is\hspace*{0.15cm}not\hspace*{0.15cm}done:\hspace*{0.15cm}  k^i_j \rightarrow   k^i_j \hspace*{0.15cm} - \hspace*{0.15cm} k_{DEC}
\label{eqn:k-dec}
\end{equation}

\subsection{A Robotic Interpretation of AFM}
\label{afm:mrs-interpretation}
The interpretation of AFM in a multi-robot system follows the above mentioned generic interpretation.  Each robot is modelled as an agent and each task is modelled as a spatial location.  The robots repeatedly select tasks and if the robot is outside a fixed task boundary, it navigates towards the task.  If the robot is within the task boundary it remains there until the end of the time step when a new (or the same) task is selected.
The distance between a task and a robot is simply the physical distance and the sensitivities are recorded as specific values on each robot.
The urgency values of the tasks are calculated based on the number of robots attending each task and the updated urgency values are communicated to the robots.

The sensing of the distance between the tasks and robots as well as the communication of urgency values are non-trivial in a robotic system.  Both the sensing and communication can be done either locally by the individual robots or centrally, through an overhead camera and a global communication network.
%==================================================================
\section{Communication in biological social systems}
\label{bio-comm}
Communication plays a central role in self-regulated DOL of biological social systems. In this section, communication among  social insects are briefly reviewed.
%
\subsection{Purposes, modalities and ranges}
Communication in biological societies serves many closely related social purposes. Most P2P communication include: recruitment to a new food source or nest site, exchange of food particles, recognition of individuals, simple attraction, grooming, sexual communication. In addition to that colony-level broadcast communication include: alarm signal, territorial and home range signals and nest markers \citep{Holldobler1990}.

Biological social insects use different modalities to establish social communication, such as, sound, vision, chemical, tactile,  electric and so forth.  Sound waves can travel a long distance and thus they are suitable for advertising signals. They are also best for transmitting complicated information quickly \citep{Slater1986}. Visual signals can travel more rapidly than sound, but they are limited by the physical size or line of sight of an animal. They also do not travel around obstacles. Thus they are suitable for short-distance private signals.\\

In ants and some other social insects, chemical communication is predominant \citep{Holldobler1990}. A pheromone is a chemical substance, usually a type of glandular secretion, used for communication within species. One individual releases it as a signal and others respond to it after tasting or smelling. Using pheromones individuals can code quite complicated messages in smells. If wind and other conditions are favourable,  this type of signals emitted by such a tiny species can be detected from several kilometres away. Thus chemical signals are extremely economical of their production and transmission. But they are quite slow to diffuse away. But ants and other social insects manage to create sequential and compound messages either by a graded reaction of different concentrations of same substance or by blends of signals.

Tactile communication is also widely observed in ants and other species typically by using their body antennae and forelegs. It is observed that in ants touch is primarily used  for receiving information rather than informing something. It is usually found as an invitation behaviour in worker recruitment process. When an ant intends to recruit a nest-mate for foraging or other tasks it runs upto a nest-mate and beats her body very lightly with  antennae and forelegs. The recruiter then runs to a recently laid pheromone trail or lays a new one. In underwater environment some fishes and other species also communicate through electric signals where their nerves and muscles work as batteries. They use continuous or intermittent pulses with  different frequencies to learn about environment and to convey their identity and aggression messages.
%%
%%%%%%%%%%%%%%%%%%%%%%%%%%%%%%%%%%%%%%%%%%
\subsection{Signal active space and locality}
%bio-comm-ants-active-space
\begin{figure}
\centering
\includegraphics[width=10cm, angle=0]
{./images/bio-comm-ants-active-space.eps}
%figure caption is below the figure
\caption{Pheromone active space observed in ants, reproduced from \protect\cite{Holldobler1990}.}
\label{fig:ants-active-space} % Give a unique label
\end{figure}
The concept of active space (AS) is widely used to describe the propagation of signals by species. In a network environment of signal emitters and receivers, active space is defined as the area encompassed by the signal during the course of transmission \citep{Mcgregor2000}. The concept of active space is described somewhat differently in case some social insects. In case of ants, this active space is defined as a zone within which the concentration of pheromone (or any other behaviourally active chemical substances) is at or above threshold concentration \citep{Holldobler1990}.
  
Fig. \ref{fig:ants-active-space} shows the use of active spaces of two species of ants: (a) {\em Atta texana} and (b) {\em Myrmicaria eumenoides}.  The former one uses two different concentrations of {\em 4-methyl-3-heptanone} to create attraction and alarm signals, whereas the latter one uses two different chemicals: {\em Beta-pinene} and {\em Limonene} to create similar kinds signals, i.e. alerting and circling.
 According to need, individuals regulate their active space by making it large or small, or by reaching their maximum radius quickly or slowly, or by enduring briefly or for a long period of time. From the precise study of pheromones it has been found that active space of alarm signal is consists of a concentric pair of hemispheres (Fig. \ref{fig:ants-active-space}). As an ant enters the outer zone, she is attracted inward toward the point source; when she next crosses into the central hemisphere she become alarmed. It is also observed that ants can release pheromones with different active spaces.
%%
Active space has strong role in modulating the behaviours of ants. For example, when workers of {\em Acanthomyops claviger} ants produce alarm signal due to an attack by a rival or insect predator, workers sitting a few millimetres away begin to react within seconds. However, those ants sitting a few centimetres away take a minute or longer to react. In many cases, ants and other social insects exhibit modulatory communication within their active space where many individuals involve in many different tasks. For example, while retrieving the large prey, workers of {\em Aphaeonogerter} ants produce chirping sounds (known as \textit{stridulate}) along with releasing poison gland pheromones. These sounds attract more workers and keep them within the vicinity of the dead prey to protect it from their competitors. This communication amplification behaviour can increase the active space to a maximum distance of 2 meters.

%%%%%%%%%%%%%%%%%%%%%
\subsection{Common communication strategies}
\label{bg:bio-comm:strategies}
%%%
%%
In biological social systems, we can find all different sorts of communication strategies ranging from indirect pheromone trail laying to local and global broadcast of various signals. The most common four communication strategies are indirect, P2P, local and global broadcast communication strategies. The pheromone trail laying is one of the most discussed indirect communication strategy among various species of ants. Fig. \ref{fig:ant-indirect} shows a pheromone trail following of a group of foraging ants. This indirect communication strategy effectively helps ants to find a better food source among multiple sources, find shorter distance to a food source, marking nest site and move there etc. \citep{Hughes2008}. Direct P2P communication strategy is also very common among most of the biological species. Fig. \ref{fig:bees-ants-p2p-comm} shows P2P communication of ants and honey-bees. This tactile form of communication is very effective to exchange food item, flower nectar with each-other or this can be useful even in recruiting nest-mates to a new food source or nest-site.
%%%%%%%%%%%%%%%%%%%%%%
\begin{figure}
\centering
\subfigure[Two honey-bees]
{
\includegraphics[width=6cm, height=4cm]{./photos/honey-bee-p2p-hy23.eps}
\hspace{0.25cm}
\includegraphics[width=6cm, height=4cm]{./photos/ants-p2p-hy14.eps}
}
\caption{Example of P2P tactile communication: (a) Honey-bees exchange nectar samples by close contact (b) ants also exchange food or information via tactile communication. \protect\newline  From http://www.harunyahya.com/.} % last seen 01/05/2010.}
\label{fig:bees-ants-p2p-comm}
\end{figure}
%%
%%-----------------------------------------------------
\subsection{Roles of communication in task-allocation}
\label{bg:bio-comm:comm-role}
%%
\begin{figure}[htp]
\centering
\includegraphics[width=5cm, angle=-90]
{./images/honey-bee-dance-stat.eps}
%figure caption is below the figure
\caption{Self-regulation in honey-bee's dance communication behaviours, produced after the results of \protect\cite{Von1967} honey-bee round-dance experiment performed on 24 August 1962.}
\label{fig:honey-bee-dance-stat}  % Give a unique label
\end{figure}
Communication among nest-mates and sensing of tasks are the integral parts of the self-regulated DOL process in biological social systems. They create necessary  preconditions for switching from one tasks to another or to attend dynamic urgent tasks. Suitable communication strategies favour individuals to select a better tasks. For example, \cite{Garnier+2007} reported two worker-recruitment experiments on black garden ants and honey-bees. The scout ants of {\em Lasius niger}  recruit uninformed ants to food source using a well-laid pheromone trails. {\em Apis mellifera} honey-bees also recruit nest-mates to newly discovered distant flower sources through waggle-dances. In the experiments,  poor food sources were given first to both ants and honey-bees. After some time,  rich food sources were introduced  to them. It was found that only honey-bees were able to switch from poor source to a rich source using their sophisticated dance communication.
%%
\begin{table}[htp]
\caption{Self-regulation of communication behaviours in biological social systems}
\label{table:bio-comm-task-urgency}
\begin{center}
\begin{tabular}{|l|l|l|}
\hline \textbf{Example event} & \textbf{Strategy} & \textbf{Modulation of communication}\\
&  &  \textbf{upon sensing tasks}\\
\hline Ant's alarm signal &  Global  & High concentration of pheromones\\
by pheromones & broadcast &  increase aggressive alarm-behaviours \\                                                                                                                                               
\hline Honey-bee's  & Local  &  High quality of nectar source increases \\
round dance & broadcast & dancing and foraging bees\\
\hline Ant's tandem run     & P2P & High quality of nest \\
for nest selection & &   increases traffic flow\\
\hline Ant's pheromone   & Indirect & Food source located at shorter distance\\
trail-laying to   & &  gets higher priority as less pheromone \\
food sources & & evaporates and more ants joins\\
\hline
\end{tabular}
\end{center}
\end{table}
%%%%%%%%%%%%%%%%%
Table \ref{table:bio-comm-task-urgency} presents the link between sensing the task and self-regulation of communication behaviours among ants and honey-bees. Here, we can see that communication is modulated based on the perception of  task-urgency irrespective of the communication strategy of a particular species. Under indirect communication strategy of ants, i.e. pheromone trail-laying, we can see that the principles of self-organization, e.g. positive and negative feedbacks take place due to the presence of different amount of pheromones for different time periods. Initially, food source located at shorter distance gets relatively more ants  as the ants take less time to return nest. So, more pheromone deposits can be found in this path as a result of positive feedback process.  Thus, the density of pheromones or the strength of indirect communication link reinforces ants to follow this particular trail.
%%
Similarly, perception of task-urgency influences the P2P and broadcast communication strategies. {\em Leptothorax albipennis} ant take lees time in assessing a relatively better nest site and quickly return home to recruit its nest-mates \citep{Pratt+2002}. Here, the quality of nest directly influences its intent to make more ``tandem-runs'' or to do tactile communication with nest-mates. We have already discussed about the influences of the quality of  flower sources to honey-bee dance.  Fig. \ref{fig:honey-bee-dance-stat} shows this phenomena more vividly. It has been plotted using the data from the honey-bee round-dance experiments of \cite[p. 45]{Von1967}. In this plot, Y1 line refers to the concentration of sugar solution. This solution was kept in a bowl  to attract honey-bees and the amount of this solution was varied from $\frac{3}{16}$M to 2M (taken as 100\%). In this experiment, the variation of this control parameter influenced honey-bees' communication behaviours while producing an excellent self-regulated DOL.

In Fig. \ref{fig:honey-bee-dance-stat} Y2 line represents the number of collector bees that return home. The total number of collectors was 55 (taken as 100\%). Y3 line plots the percent of collectors displaying round dances. We can see that the fraction of dancing collectors is directly proportional to the concentration of sugar solution or the sensing of task-urgency. Similarly, the average duration of dance per bee  is plotted in Y4 line. The maximum dancing period was 23.8s (taken as 100\%). Finally, from Y5 line we can see the outcome of the round-dance communication as the number of newly recruited bees to the feeding place. The maximum number of recruited bees was 18 (taken as 100\%). So, from an overall observation, we can see that bees sense the concentration of food-source  as the task-urgency and they self-regulate their round-dance communication behaviour according to their perception of this task-urgency. Thus, this self-regulated dancing behaviour of honey-bees attracts an optimal number of inactive bees to work.

Broadcast communication is one of the classic ways to handle dynamic and urgent tasks in biological social systems. It can be commonly observed in birds, ants, bees and many other species. Table \ref{table:bio-comm-task-urgency} mentions about the alarm communication of ants. Similar to the honey-bee's dance communication, ants has a rich language of chemical communication that can produce words through blending of different glandular secretions in different concentrations. Fig. \ref{fig:ants-active-space} shows how ants can use different concentrations of chemicals to make different stimulus for other ants. From the study of ants, it is clear to us that taking defensive actions, upon sensing a danger, is one of the highest-priority tasks in an ant colony. Thus, for this highly urgent task, ants almost always use their global broadcast communication strategy through their strong chemical signals and they make sure all individuals can hear about this task.  This gives us a coherent picture of the self-regulation of biological species based on their perception of task-urgency.
%%-----------------------------------------------
\subsection{Effect of group size on communication}
\label{bg:bio-comm:group-size}
The performance of cooperative tasks in large group of individuals also depends on the communication and sensing strategies adopted by the group. Depending on the group size, different kinds of information flow occur in different types of social wasps \citep{Jeanne1999}. Polistes independent founders  are species in which reproductive females establish colonies alone or in small groups with about $10^2$ individuals at maturity. Polybia swarm founders initiates colonies by swarm of workers and queens. They have a large number of individuals, in the order of $10^6$ and 20\% of them can be queen. In case of swarm founders information about nest-construction or broods food-demand can not reach to foragers directly.  Among the swarm founders for nest construction. The works of {\em pulp foragers} and {\em water foragers} depend largely on their communication with {\em builders}. On the other hand, in case of independent founders there is no such communication and sensing are present among individuals.

The above interesting findings from social wasps have been linked up with  the group productivity of wasps. \cite{Jeanne1999} reported high group productivity in case of LSLC of swarm founders. The per capita productivity was measured as the number of cells built in the nest and the weight of dry brood in grams. In case of independent founders this productivity is much lesser (max. 24 cells per queen at the time the first offspring observed) comparing to the thousands of cells produced by swarm founders.  This shows  us the direct link between high productivity of social wasps and their selection of LSLC strategy. These fascinating findings from wasp colonies have motivated us to test these communication and sensing strategies in a fairly large multi-robot system to achieve an effective self-regulated MRTA.
%==============================================================
\section{Communication in multi-robot systems}
\label{mrs-comm}
\label{bg:mrs-comm}
Communication plays an important role for any high-level interaction e.g. cooperation among a multi-robot team. Below we have described the dominant issues of communication among multi-robot teams with a focus on how communication  can lead to produce effective MRTA solutions.
%-----------
\subsection{Rationale of communication}
Most of the robotic researchers generally agree that communication in multi-robot system usually provides several major benefits. Robots can exchange potential information based on their spatial position and knowledge of past events. This, in turn, leads to improve perception over a distributed region without directly sensing it. In order to perform (or stop performing) certain tasks simultaneously or in a particular order, robots need to communicate, or signal, to each other. Communication can help a lot to influence each-other in a team that, in turn, enables robots to interact and negotiate their actions effectively.
%----------------- 
\subsection{Information content}
Although communication provides several benefits for team-work it is costly to provide communication support in terms of hardware, firmware as well as run-time energy spent in communication. So robotic researchers carefully minimize the necessary information content in communications by using suitable communication protocols and high-level abstractions.  The potential information contents that can be used in communication among robots are mainly:\\
%%
\textbf{- Individual state:} ID number, battery level, task-performance statistics, etc.\\
\textbf{- Goal:} Location of target task or all tasks discovered.\\
\textbf{- Task-related state:} The amount of task completed.\\
\textbf{- Environmental state:} Free and blocked paths, level of interference found, any urgent event or dangerous changes found in the environment.\\
\textbf{- Intentions:} Detail plan for doing a task or sequences of selected actions.
%% 
%------------------------------------------
\subsection{Communication modalities}
Robotic researchers typically use robot's on-board wireless radio, infra-red (IR), vision and sound hardware modules for robot-robot and robot-host communication. The reduction in price of wireless radio hardware chips e.g. wifi (ad-hoc WLAN 802.11 network) or Bluetooth makes it possible to use wireless radio communication widely. Inexpensive IR communication module is also typically built into almost all mobile robots due to its low-cost and suitability for ambient light and obstacle detection. IR can also be used for low bandwidth communication in short-ranges, e.g. keen-recognition. Most robots can also produce basic sound waves and detect it with their built-in speakers and on-board microphones. 
%-----------------------------
\subsection{Communication strategies}
%\begin{figure}
%\centering
%\includegraphics[width=6cm, angle=0]
%{./images/mrs-comm-complexities.eps}
%%figure caption is below the figure
%\caption{Three aspects of communication in multi-robot system}
%\label{fig:mrs-comm-strategies} % Give a unique label
%\end{figure}
Whatever be the communication modalities of a multi-robot system, suitable strategies are required to disseminate information in a timely manner to a target audience that maximizes the effective task-completion and minimizes delays and conflicts. The complexities of communication strategies can be elaborated in terms of three independent aspects: organization, expressiveness and range of communication. These are described below.
%%
\subsection*{Organization of communication structure}
Communication in a multi-robot system can be organized using an external/internal central entity (e.g. a server PC, or a leader robot) or, a few leader robots, or by using decentralized or local schemes where every robot has the option to communicate with every other robot of the team. From a recent study of multi-robot flocking \cite{Celikkanat+2008} have shown that a mobile robot flock can be steered toward a desired direction through externally guiding some of its members, i.e. the flock relies on multiple leaders or information repositories. Note that here task-allocation is fully decentralized i.e. each robot selects its task, but the communication structure is hybrid; robots communicate with each other and with a centralized entity.
%%--------------------------------
\subsection*{Expressiveness of communication}
Communication in a multi-robot system can also be characterized its expressiveness or the degree of explicitness. In one extreme it can be fully implicit, e.g. stigmergic, or on the other end, it can be fully explicit where communication is done by a rich vocabulary of symbols and meanings. Researchers generally tend to stay in either end based on the robotic architecture and task-allocation mechanism used. However, both of these approaches can be tied together under any specific application. They are highlighted below.
%%
\begin{enumerate}
\item \textbf{Explicit or direct communication: }
This is also known as intentional communication. This is done purposefully by usually using suitable modality e.g. wireless radio, sound, LEDs. Because explicit communication is costly in terms of both hardware and software, robotic researchers always put extra attention to design such a system by analysing strict requirements such as communication necessity, range, content, reliability of communication channel (loss of message) etc.
%
\item \textbf{Implicit or indirect communication:} 
This is also known as indirect stigmergic communication. This is a powerful way of communication where individuals leave information in the environment. This method was adopted from the social insect behaviour, such as stigmergy of ants (leaving of small amount of pheromone or chemicals behind while moving in a trail).
\end{enumerate}
%%
%%--------------------------
\subsection*{Target recipients of message}
The target recipient selection or determining the communication range or sometimes called radius of communication is an interesting issue in multi-robot system research. Researchers generally tries to maximize the information gain by using larger range. However, transmission power and communication interference among robots play a major role to limit this range. The following major instances of this strategy can be used.
\begin{itemize}
\item \textbf{Global broadcast:} where all robots in the team can receive the message.
\item \textbf{Local broadcast:} where a few robots in local neighbourhood can receive the message.
\item \textbf{Publish-subscribe:} where only the  subscribed robots can receive the message.
\item \textbf{Peer-to-peer:} where only the closest peer robot can receive the message.
\end{itemize}
%----------------------------------------------------------
%\subsection*{Determination of local neighbourhood in multi-robot communication}
%Most researchers in the area of swarm-robotic system, who use algorithms based on local-neighbourhood of communication, face this problem of defining the range of local neighbourhood. \cite{Agah+1995} presented that larger communication range is not always optimum for some types of tasks e.g. exploration where a large number of recipient robots decreased the performance of exploration task. \cite{Yoshida+2000} provided a design of optimal communication range of homogeneous robots based on their spatial and temporal analyses of information diffusion within the context of cooperative tasks in a manufacturing shop-floor. Spatial design tried to minimize the time for information transmission and temporal design tried to minimize the information announcing time to avoid excessive information diffusion. Eq. \ref{egn:yoshida-range} describes their optimal range $\chi_{optimal}$ as a function of information acquisition capacity of robots ($c$) and the probability of information output of a robot ($p$). Here $c$ is an integer representing the upper-limit of the number of robots that can be the target recipients at any time without the loss of information and $\chi_{optimal}$ gives the average number of robots within the output range.
%\begin{equation}
%\chi_{optimal} = \frac{\sqrt [c] {c!}}{p}
%\label{egn:yoshida-range}
%\end{equation}
%%%
%%-------------------------------------------------
\subsection{Role of communication in MRTA}
Although researchers in the field of multi-robot system have been adopting various communication strategies for achieving MRTA solutions in different task domains, very few studies correlate the role of communication with the effectiveness of MRTA. This is due to the fact that researchers usually adopt a certain task-allocation method and they limit their use of communication strategy to either explicit global/local broadcast (in most predefined task-allocation researches) or  implicit/no communication (in most self-organized task-allocation researches). Here we have attempted to scrutinize how MRTA solutions have been affected by the variations in communication strategies.

\cite{kalra+2007} empirically studied the comparative performance of MRTA under both predefined and self-organized approaches with event-driven simulations. They found that the accuracy of information is crucial for predefined market-based approach where every robot communicated with every other robot. In case of unreliable link, threshold-based approach performed same as market-based approach, but with less computational overhead. In case of varying robot's communication range,  they found that market-based approach performed well for a short communication range where robots were able to communicate with less than a third of the total number of team-mates. 
  
In order to pursue MRTA, robots can receive information from a centralised source \cite{Krieger+2000} or from their local peers \citep{Agassounon+2002}. This centralized communication system is easy to implement. However as we mentioned before, this system has disadvantage of a single point of failure and it is not scalable. On the other hand, uncontrolled reception of information from decentralized or local sources is also not free from drawbacks. If a robot exchanges signals with all other robots, it might get the global view of the system quickly and can select an optimal or near optimal task. This can produce a great improvement in overall performance of some types of tasks e.g., in area coverage \cite{Rutishauser+2009}. But this is also neither practical nor scalable for a typically large multi-robot system.

A potential alternate solution of this problem can be obtained by decreasing the number of message recipients on the basis of a local communication range. This means that robots are allowed to communicate only with those peers who are physically located within a pre-set distance. When this strategy is used for sharing task information among peers, MRTA can be more robust to the dynamic changes in the environment and energy- efficient \citep{Agassounon+2002}. Simailar to this, \cite{Pugh+2009} reported a distributed multi-robot learning scenario with two cases: 1) robots were allowed to communicate with any two other robots ({\em Model A}) and 2)  robots were allowed to communicate with all robots in a fixed radius ({\em Model B}). In simulation and real robotic experiments with 10 robots and communication ranges of 0.3 m, 1.0 m and 3.3 m, they showed that Model B performed better in intermediate communication range. 

Many robotic researchers tried to use some forms of adaptation rules in local communication to avoid saturation of the communication channel, e.g. based on robot densities in a given area. \cite{Yoshida+2000} tried to formalize the suitable communication range based on spatial and temporal properties of information diffusion of a given communication channel. The major focus of this type of research is to measure the cost of communication based on some metrics, e.g.  transmission time and collisions with other robots, and then regulate communication ranges dynamically. These ideas are attractive to maximize information gain in dynamic environment, but there is no point of doing communication if there is little or no task-requirement. 

\cite{Oca+2005} acknowledged the above fact within the context of their ant-based clustering experiments. They used two simple communication strategies: 1) simple memory sharing by robots (shared memory access) and 2) shared use of environment maps (global sensing). In both of these cases, it was found that communication is only useful when some initial random clustering phase was passed. The accuracy of shared information in highly dynamic environment was poor and did not carry any significant advantage. In case of local memory sharing by robots, they showed that  sharing information within a limited number of robots produced more efficient  clusters, rather than  not sharing information at all in stigmergic communication mode. However, sharing memory in a large group is not a feasible communication strategy because of the huge latencies and interferences involved in the communication channel.
%=====================================================================
\section{Validation of AFM under centralized communication strategy}
In this section, in order to present the validation of AFM, we first describe our manufacturing shop floor scenario and then the centralized communication model along with its implementation under this scenario. Finally we present the experimental results that validates our model.
\subsection{A manufacturing shop-floor scenario}
\label{validation:scenario}
By extending our interpretation of AFM in multi-robot system, we can set-up manufacturing shop-floor  scenario. Here, each task represents a manufacturing machine that is  capable of producing goods from raw materials, but they also require constant maintenance works for stable operations. Let $W_{j}$ be a finite number of material parts that can be loaded into a machine $j$ in the beginning of its production process and in each time-step, $\omega_{j}$ units of material parts can be processed  ($\omega_{j} \ll W_{j} $). So let $\Omega_{j}^{p}$ be the initial production workload of $j$ which is simply: $W_{j} / \omega_{j}$ unit.
%
We assume that all machines are identical. In each time step, each machine always requires a minimum threshold number of robots, called hereafter as {\em minimum robots per machine ($\mu$)}, to meet its constant maintenance work-load, $\Omega_{j}^{m}$ unit. However, if $\mu$ or more robots are present in a machine for production purpose, we assume that, no extra robot is required to do its maintenance work separately. These robots, along with their production jobs, can do necessary maintenance works concurrently. For the sake of simplicity, here we consider $\mu$ = 1.

Now let us fit the above production and maintenance work-loads and task performance of robots into a unit task-urgency scale. Let us divide our manufacturing operation into two subsequent stages: 1) \acfi{PMM}, and 2) \acfi{MOM}. Initially a machine starts working in PMM and does production and maintenance works concurrently. When there is no production work left, then it  enters into MOM. Fig. \ref{fig:vsp} illustrates this scenario for a single machine.

Under both modes, let $\alpha_{j}$ be the amount of workload occurs in a unit time-step if no robot serves a task and it corresponds to a fixed task-urgency $\Delta \phi_{INC}$. On the other hand, let us assume that in each time-step, a robot, $i$, can decrease a constant workload $\beta_{i}$ by doing some maintenance work along with doing any available production work. This  corresponds to a negative task urgency: $- \Delta \phi_{DEC}$. So, at the beginning of production process, task-urgency, occurred in a machine due to its production work-loads, can be encoded by Eq. \ref{eqn:task-urgency-prod-init}.
\begin{equation}
\small
\Phi_{j, INIT}^{PMM} = \Omega_{j}^{p} \times \Delta \phi_{INC} + \phi_{j}^{m0}
\label{eqn:task-urgency-prod-init}
\end{equation}
where $\phi_{j}^{m0}$ represents the task-urgency due to any initial maintenance work-load of $j$.
Now if no robot attends to serve a machine, each time-step a constant maintenance workload of $\alpha_{j}^{m}$ will be added to $j$ and that will increase its task-urgency by $\Delta \phi_{INC}$. So, if $k$ time steps passes without any production work being done, task urgency at $k^{th}$ time-step will follow Eq. \ref{eqn:task-urgency-prod-case1}.
\begin{equation}
\small
\Phi_{j, k}^{PMM} =\Phi_{j, INIT}^{PMM} + k \times \Delta \phi_{INC}
\label{eqn:task-urgency-prod-case1}
\end{equation}
However, if a robot attends to a machine and does some production works from it, there would be no extra maintenance work as we have assumed that $\mu$ = 1. Rather, the task-urgency on this machine will decrease by $\Delta \phi_{DEC}$ amount. If $\nu_{k}$ robots work on a machine simultaneously at time-step $k$, this decrease will be: $\nu_{k} \times \Delta \phi_{DEC}$. So in such cases, task-urgency in $(k+1)^{th}$ time-step can be represented by:
\begin{equation}
\small
\Phi_{j, k+1}^{PMM} = \Phi_{j, k}^{PMM} - \nu_{k} \times \Delta \phi_{DEC}
\label{eqn:task-urgency-prod-case2}
\end{equation}
At a particular machine $j$, once $\Phi_{j, k}^{PMM}$ reaches to zero, we can say that there is no more production work left and this time-step $k$ can give us the {\em production completion time} of $j$, $T_{j}^{PMM}$. Average production time-steps of a shop-floor with M machines can be calculated by the following simple equation.
\begin{equation}
\small
T_{avg}^{PMM} = \frac{1}{M} \sum_{j=1}^{M} T_{j}^{PMM} 
\label{eqn:avg-pmm}
\end{equation}
$T_{avg}^{PMM}$ can be compared with the minimum number of time-steps necessary to finish production works, $T_{min}^{PMM}$. This can only happen in an ideal case where all robots work for production without any random walking or failure. We can get $T_{min}^{PMM}$ from the total amount of work load and maximum possible inputs from all robots. If there are M machines and N robots, each machine has $\Phi_{INIT}^{PMM}$ task-urgency, and each time-step robots can decrease N $\times$ $\Delta \phi_{DEC}$ task-urgencies, then the theoretical $T_{min}^{PMM}$ can be found from the following Eq. \ref{eqn:min-pmm}.
%
\begin{multicols}{2}
\small
\begin{equation}
T_{min}^{PMM} = \frac{M \times \Phi_{INIT}^{PMM}}{N \times \Delta \phi_{DEC}} 
\label{eqn:min-pmm}
\end{equation}
\vspace*{0.2cm}
\begin{equation}
\zeta_{avg}^{PMM} = \frac{T_{avg}^{PMM} - T_{min}^{PMM}}{T_{min}^{PMM}} 
\label{eqn:appd}
\end{equation}
\end{multicols}
Thus we can define $\zeta_{avg}^{PMM}$, \acf{APCD} by following Eq. \ref{eqn:appd}:
%%
When a machine enters into MOM, only $\mu$ robots are required to do its maintenance works in each time step. So, in such cases, if no robot serves a machine, the growth of task-urgency will follow Eq. \ref{eqn:task-urgency-prod-case1}. However, if $\nu_{k}$ robots are serving this machine at a particular time-step $k^{th}$ , task-urgency at $(k+1)^{th}$ time-step can be represented by:
\begin{equation}
\small
\Phi_{j, k+1}^{MOM} = \Phi_{j, k}^{MOM}- (\nu_{k} - \mu) \times \Delta \phi_{DEC}
\label{eqn:task-urgency-maint-case}
\end{equation}
By considering $\mu = 1$, Eq. \ref{eqn:task-urgency-maint-case} will reduces to Eq. \ref{eqn:task-urgency-prod-case2}. Here, $\Phi_{j, k+1}^{MOM}$ will correspond to the {\em pending maintenance work-load} of a particular machine at a given time. This happens due to the random task switching of robots with a no-task option (random-walking). Interestingly PMW will indicate the robustness of this system since higher PMW value will indicate the delay in attending maintenance works by robots. We can find the \acfi{APMW} per time-step per machine, $\chi_{j}^{MOM}$ (Eq. \ref{eqn:sigle-pmw}) and average PMW per machine per time-step, $\chi_{avg}^{MOM}$ (Eq. \ref{eqn:avg-pmw}).
\begin{multicols}{2}
\small
\begin{equation}
\chi_{j}^{MOM}= \frac{1}{K} \sum_{k=1}^{K} \Phi_{j, k}^{MOM}
\label{eqn:sigle-pmw}
\end{equation}
\vspace*{0.2cm}
\begin{equation}
\chi_{avg}^{MOM}= \frac{1}{M} \sum_{j=1}^{M} {\chi_{j}^{MOM}}
\label{eqn:avg-pmw}
\end{equation}
\end{multicols}
%----------------------------------------------------------
\subsection{Centralized communication model}
\begin{figure}
\centering
\includegraphics[height=5cm, angle=0]{./images/CentralizedComm.eps}
\caption{\small A centralized communication scheme} % for implementing AFM}
\label{fig:ccm} % Give a unique label
\end{figure}
%%
AFM relies upon a system-wide continuous flow of information which can be realized using any suitable communication model. A simple centralized communication scheme is outlined in Fig. \ref{fig:ccm}. In this model we have used bi-directional signal-based exchange of communication messages between a centralized \textit{task perception server} (TPS) and a \textit{robot-controller client} (RCC). The main role of TPS is to send up-to-date task-information to RCCs. This  task-information mainly contains the location and urgency of all tasks  which is used by the RCCs for running their task-allocation algorithm. The urgency value of each task is dynamically updated  by TPS after receiving the  status signals from the working robots of that particular task. Fig. \ref{fig:ccm} shows how three robots are attracted to two different tasks and their communications with TPS.
%%
We can characterize our communication model in terms of three fundamental issues of communication: 
i) message content, ii) communication frequency and iii) target message recipients \citep{Gerkey+2001}.
%%
AFM suggests the communication of task-urgencies  among robots. This communication helps the robots to gain information that can be  treated as ``global sensing''. However in this model  robots do not communicate among themselves. Hence this model can be approximated as the GSNC strategy. Since in order to run the task-allocation algorithm robot-controllers need the distance information we also include the task position information in  the message. Our centralized communication model is open to include any further information, such as time-stamp, in the message payload. In this model the frequency of signal emission depends on several issues, e.g. the rate at which the environment is changing, the bandwidth of communication medium. In case of time-extended tasks, robots can receive information less frequently and the  bandwidth usage can be kept  minimum. However under a fast changing environment relatively more bandwidth will be required.  Finally the centralized communication model spread the attractive fields of all tasks globally by broadcasting information to all robots.  

%Our centralized communication model has been implemented by the D-Bus interprocess communication technology \cite{Pennington+2010}. Under this implementation, we have used D-Bus \textit{signal} type asynchronous messages to enable information sharing among SwisTrack multi-robot tracker \cite{Lochmatter+2008}, TPS and RCCs inside a single host. D-Bus signals give us the flexible, fault-tolerant and real-time messaging scheme which can not be easily achieved in other interprocess communication schemes. The detail design and implementation  of our of centralized communication model can be found in \cite{Sarker2010control}. 
%-----------------------------------------------------------
\subsection{Experiment design}
We have designed a set of  manufacturing shop-floor scenario experiments for validating the effectiveness of our AFM in producing self-regulated MRTA. In our experiments we design the following observables.\\
\textbf{- Plasticity:} Within our manufacturing shop-floor context, plasticity refers to the collective ability of the robots to switch from doing no-task option (random-walking) to doing a task (or vice-versa) depending on the work-load present in the system. Here we expect to see that most of the robots would be able to engage in tasks when there would be high workloads (or task-urgencies). The changes of task-urgencies and the ratio of robots engaged in tasks can be good metrics to observe plasticity in MRTA.

\textbf{- Task-specialization:} Self-regulated MRTA is generally accompanied with task-specializations of agents. That means that few robots will be more active than others. From the interpretation of AFM, we can see that after doing a task a few times, a robot will soon be sensitized to it. Therefore, from the raw log of task-sensitization of robots, we can be able to find the pattern of task-sensitization of robots per task basis.

\textbf{- Quality of task-performance:} As discussed in Sec. \ref{validation:scenario} we can measure the quality of MRTA from the APCD. It first calculates the ideal minimum production time and then finds the delay in production process from the actual production completion data. Thus this will indicate how much more time is  spent in the production process due to the self-regulation of robots.

\textbf{- Robustness:} In order to see if our system can respond to the gradually increasing workloads,  we can measure APMW within the context of our manufacturing shop-floor scenario. This can show the robustness of our system. When a task is not being served by any robot for some time we can see that its urgency will rise and robots will respond to this dynamic demand. 

\textbf{- Flexibility:} From the design of AFM, we know that robots that are not doing a task will be de-sensitized to it or forget that task. So at an overall low work-load (or task urgency), less robots will do the tasks and hence less robots will have the opportunity to learn tasks. From the shop-floor work-load data, we can confirm the presence of flexibility in MRTA.

\textbf{- Energy-efficiency:} In order to characterize the energy-efficiency in MRTA we can log the pose data of each robot that can give us the total translations occurred by all robots in our experiments. This can give us a rough indication of energy-usage by our robots. 

\textbf{- Information flow:} Since AFM requires a system-wide continuous flow of information, we can measure the communication load to bench-mark our implementation of communication system. This bench-mark data can be used to compare among various communication strategies. Here we can measure  how much task-related information, i.e. task-urgency, location etc. are sent to the robots at each time step. %This  amount of information or communication load can be constant or variable depending on the design of the communication system.

\textbf{- Scalability:} In order to see the effects of scaling on MRTA, we have designed two group of experiments. Series A corresponds to a small group where we have used 8 robots, 2 tasks under an arena of 2 $m^2$. We have doubled these numbers in other experiments, i.e. 16 robots, 4 tasks under an arena of 4 $m^2$. This proportional design can give us a valuable insight about the effects of scaling on self-regulated MRTA. 
%%
\begin{table}
\caption{Experimental parameters of Series A \& B experiments}
\label{table:params}
\begin{center}
\begin{tabular}{|p{2in}|c|}
\hline Parameter & Value\\
%\begin{tabular}{|p{2in}|c|}
%\hline Parameter & Series A $\mid$ Series B\\
%\hline Total number of robots ($N$) & \hspace*{0.1cm} 8 $\mid$ 16\\
%\hline Total number of tasks ($M$) & 2 $\mid$ 4\\
%\hline Experiment area ($A$) & 2 $m^2$ $\mid$  4 $m^2$\\
\hline Initial production work load/machine ($\Omega_{j}^{p}$) & 100 unit \\
\hline Task urgency increase rate ($\Delta\phi_{INC}$) & 0.005\\
\hline Task urgency decrease rate ($\Delta\phi_{DEC}$) & 0.0025\\
\hline Initial sensitization ($K_{INIT}$) & 0.1\\
\hline Sensitization increase rate ($\Delta k_{INC}$) & 0.03\\
\hline Sensitization decrease rate ($\Delta k_{DEC}$) & 0.01\\
\hline
\end{tabular}
\end{center}
\end{table}
%%
%------------------------------------------------------------
%\subsection{Parameters}
In order to observe the self-regulated MRTA, we have designed our experiments to record the following  observables in each time-step:\\
%\begin{enumerate}
task-urgency of each task ($\phi$),%\\
number of robots engaged in each task,%\\
task-sensitizations ($k$) of robots,%\\
pose data of robots and %\\
communication of task-information messages.
%\end{enumerate}
Table \ref{table:params} lists a set of common parameters of our  experiments. The initial values of task urgencies correspond to 100 units of production work-load without any maintenance work-load as outlined in Eq. \ref{eqn:task-urgency-prod-init}. For task-urgency (and task-sensitization) limits, we choose a limit of 0 and 1, where 0 means no urgency (complete forgetting) of a task and 1 means maximum urgency (or full specialization) of that task. We choose a initial sensitization value of 0.1 for all tasks. 
%
%-------------------------------------------------------------
\subsection{Implementation}
\begin{figure*}
\centering
\includegraphics[width=0.9\textwidth, angle=0]
{./images/RIL-Expt-Setup1.eps}
%figure caption is below the figure
\caption{Hardware and software setup}
\label{fig:setup} % Give a unique label
\end{figure*}
As shown in Fig. \ref{fig:setup}, in our implementation there exists a centralized \textit{TaskPerceptionServer} that is responsible for disseminating task information to RCCs. The contents of task information can be physical locations of tasks, their urgencies and so on. TaskPerceptionServer delivers this information by emitting \textit{TaskInfo} signals periodically. For example, in a wireless network it can be a message broadcast. TaskPerceptionServer has another interface for catching feedback signals from robots. The \textit{RobotStatus} signal can be used to inform TaskPerceptionServer about a robot's current task id, its device status and so on. TaskPerceptionServer uses this information to update relevant part of task information such as, task-urgency. This up-to-date information is sent in next TaskInfo signal.
%%
In order to track all robots in real-time, we have used SwisTrack \citep{Lochmatter+2008} with a 16-megapixel overhead GigE camera. This set-up gives us the position, heading and id of each of the robots by processing the image frames at about 1 FPS. The interaction of the hardware and software of our system is illustrated in Fig. \ref{fig:setup}.
For inter-process communication (IPC), we have used D-Bus technology \citep{Pennington+2010}. We have developed an IPC component for SwisTrack  that can broadcast id and pose of all robots in real-time over our server's D-Bus interface.

Apart from SwisTrack, we have implemented two major software modules: {\em TaskServer} and {\em RCC}. They are developed in Python with its state of the art \textit{Multiprocessing}\footnote{http://docs.python.org/library/multiprocessing.html} module. This python module simplifies our need to manage data sharing and synchronization among different sub-processes. As shown in Fig. \ref{fig:setup}, RCC consists of four sub-processes. {\em SignalListener} and {\em SignalEmitter}, interface with SwisTrack D-Bus Server and TaskServer respectively. {\em TaskSelector} implements AFM algorithms for task selection . {\em DeviceController} moves a robot to a target task. Bluetooth radio link is used as a communication medium between a RCC and a corresponding e-puck robot. The detail design and implementation  of our of centralized communication model can be found in \citep{Sarker2010control}. 
%----------------------------------------------------------------
\subsection{Results and discussions}
\label{validation:results}
Our AFM validation experiments were conducted with 8 and 16 robots, 4 tasks in an arena of 4 $m^2$ for about 40 minutes and averaged them over five iterations. For the sake of brevity, here we describe only the results of Series B experiments (with 16 robots).

In our experiments we have defined shop-floor work-load in terms of task urgencies. For example, Eq. \ref{eqn:task-urgency-prod-init} shows how we have calculated initial production work-load of our manufacturing shop-floor scenario. Fig. \ref{fig:raw-urgencies} shows the dynamic changes in task urgencies in one iteration. The fluctuations in this plot is resulted from the different levels of task-performance of our robots.

In order to measure the task-related work-loads on our system we have summed up the changes in all task-urgencies over time. We call this as {\em shop-floor work-load history} and formalized as follows. Let $ \phi_{j, q}$ be the urgency of a task $j$ at $q^{th}$ step and $\phi_{j, q+1}$ be the task urgency of $(q+1)^{th}$ step. We can calculate the sum of changes in urgencies of all $M$ tasks at $(q+1)^{th}$ step:
\begin{equation} 
\Delta \Phi_{j, q+1} = \sum_{j=1}^{M} (\phi_{j, q+1} - \phi_{j, q})
\label{eqn:Delta-Phi}
\end{equation}
From Fig. \ref{fig:urgency-stat} shows the dynamic shop-floor workload. Here, we can see that initially the sum of changes of task urgencies (shop-floor workload) is going towards negative direction. This implies that tasks are being served by a high number of robots.  Fig. \ref{fig:worker-stat} shows that in production stage, when  work-load is high, many robots are active in tasks and this ratio varies according to task urgency changes.

To measure the task performance in our manufacturing shop-floor scenario, we have calculated the APCD and APMW as shown in Fig. \ref{fig:mfg-stat}. For Series A we have got  average production completion time 111 time-steps (555s) where sample size is 10 tasks, SD = 10 time-steps. According to Eq. \ref{eqn:min-pmm}, our theoretical minimum production completion time is 50 time-steps (250s). Eq. \ref{eqn:appd} gives us APCD, $\zeta$ = 1.22. For Series B, we have got average production completion time 165 time-steps where sample size is 20 tasks, SD = 72 time-steps and APCD, $\zeta$ = 2.3. For APMW, Series A experiments give us an average time length of 369 time-steps (1845s).  In this period we have calculated APMW = 1 time-step with SD = 1 time-step and $\Delta \Phi_{INC}$ = 0.005 per task per time-step. This shows a very low APMW ($\chi$ = 0.000235) implying a very high robustness of our system. For Series B experiments, we have got APMW, $\chi$ = 0.012756 which corresponds to the pending work of 3 time-steps where SD = 13 time-steps. This also tells us the robust task performance of our robots which can return to an abandoned task within a minute or so.
%% Specialization
We have measured the task-specialization of the robots based-on their peak value of sensitization. This maximum value represents how long a robot has repeatedly been selecting a particular task. Since tasks are homogeneous we have considered the maximum sensitization value of a robot among all tasks during an experiment run. This value is then averaged for all robots using the following  equation. 
%%
\begin{equation}
K^G_{avg} = \frac{1}{N}\sum_{i=1}^{N} \max_{j=1}^M\left ( k^i_{j, q} \right ) 
\label{eqn:K-G}
\end{equation}
%%
If a robot $r_i$ has the peak sensitization value $k^i_j$ on task $j$ ($j \in M$)  at $q^{th}$ time-step, Eq. \ref{eqn:K-G} calculates the average of the peak task-specialization values of all robots for a certain iteration of our experiments. We have also averaged the time-step values ($q$) taken to reach those peak values for all robots using the following equation.
%%
\begin{equation}
Q^G_{avg}= \frac{1}{N}\sum_{i=1}^{N} q^i_{k=k_{max}}
\label{eqn:Q-G}
\end{equation}
In Eq. \ref{eqn:Q-G}, $q^i_{k=k_{max}}$ represents the time-step of robot $r_i$  where its sensitization value $k$ reaches the peak $k_{max}$ as discussed above. By averaging this peak time-step values of all robots we can have an overall idea of how many task-execution cycles are spent to reach the maximum task-specialization value $K^G_{avg}$.
Based on Eq. \ref{eqn:K-G} and Eq. \ref{eqn:Q-G}, we have got the peak task-sensitization $K^G_{avg} 
$ values: 0.40 (SD=0.08)  and 0.30 (SD=0.03), and their respective time-step $Q^G_{avg}$ values: 38 (SD=13) and 18 (SD=5) time-step. Here the robots in Series A had higher chances of task-specialization than that of Series B experiments.
%%%
%%% Communication load %%%
Fig. \ref{fig:signal-frequency-stat} presents the frequency of signalling task information by TaskServer. Since the duration of each time step is 50s long and TaskServer emits signal in every 2.5s, there is an average of 20 signals in each time-step.
%%
\begin{figure}
\begin{minipage}[t]{0.48\linewidth}
\centering
\includegraphics[width=6cm, height=4cm, angle=0]
{images/PlotUrgencyLog-2010May10-115549.eps}
%figure caption is below the figure
\caption{\small Dynamic task-urgency changes.}
\label{fig:raw-urgencies} % Give a unique label
\end{minipage}
\hspace{0.5cm}
\begin{minipage}[t]{0.48\linewidth}
\centering
\includegraphics[width=6cm, height=4cm, angle=0]{images/TaskUrgencyStat.eps}
\caption{\small Shop-floor workload history} % measured in terms of task urgencies
\label{fig:urgency-stat} % Give a unique label
\end{minipage}
\end{figure}
%%
%%% Sensitization and Translation %%%
\begin{figure}
\begin{minipage}[t]{0.48\linewidth}
\centering
%\includegraphics[height=4.8cm, angle=0]
\includegraphics[width=6cm, height=4cm, angle=0]
{images/Global-SignalingFreqStat.eps}
%figure caption is below the figure
\caption{\small Task server's task-info broadcasts}
\label{fig:signal-frequency-stat}
%
\end{minipage}
\hspace{0.5cm}
\begin{minipage}[t]{0.48\linewidth}
\centering
%\includegraphics[height=4.8cm, angle=0]
\includegraphics[width=6cm, height=4cm, angle=0]
{images/SB-WorkerRatio.eps}
\caption{\small Self-organized task-allocation}
\label{fig:worker-stat} % Give a unique label
\end{minipage}
\end{figure}
%%
%\begin{figure}
%\begin{minipage}[t]{0.48\linewidth}
%\centering
%\includegraphics[height=4cm, angle=0]{images/TaskSpecialization-task3-10may-1.eps}
%\caption{\small Task specialization on Task3}
%\label{fig:single-robot-sensitizations} % Give a unique label
%\end{minipage} 
%%%%
 
%%% Raw urgencies - Workload
\begin{figure}
\begin{minipage}[t]{0.48\linewidth}
\centering
\includegraphics[width=6cm, height=4cm]{images/SD-PlotUrgencyLog-2010Feb17-112141.eps}
\caption{\small Task-urgencies observed in Series D experiments}
\label{fig:raw-urgencies-SD} 
\end{minipage} 
%%%
\hspace{0.5cm}
\begin{minipage}[t]{0.48\linewidth}
\centering
\includegraphics[width=6cm, height=4cm]{images/SD-TaskUrgencyStat.eps}
\caption{\small Shop-floor work-load history of Series D experiments}
\label{fig:workload-SD}  
\end{minipage}
\end{figure}
%%
From the above results, we have noted several aspects of self-regulated MRTA that expose the effectiveness of AFM. As we have pointed out that this self-regulated MRTA, as observed in biological and human social systems, needs to satisfy several important characteristics, particularly plasticity and task-specialization. In addition to satisfying those basic qualities, AFM has demonstrated many other aspects. Our self-regulated robots, driven by AFM, effectively handle the dynamic work-load in our manufacturing shop-floor. They can dynamically support the need to work on demanding tasks, if there any. The variations of active worker ratio shows this.
We have observed the effect of scaling-up the robot team size. The system size of Series B is double of that of Series A in terms of robots, tasks and experiment arena. Keeping a fixed ratio of robot-to-task and task-to-arena we have intended to see the scaling effects in our experiments. Here we see that both systems can show sufficient self-regulated DOL, but task-performance of both systems varies significantly. For example, the value of APCD in Series B is higher by 1.08. This means that performance  is decreased in Series B experiments despite having the resources in same proportion in both systems. 
%======================================================================
\section{Comparisons between local and centralized communication strategies}
In most swarm robotic research local communication is considered as the one of the most critical components of the swarms where the global behaviours emerges from the local interactions among the individuals and their environment. In this study, we have used the concepts of pheromone active-space of ants to realize our simple LSLC scheme. Ants use various chemical pheromones with different active spaces (or communication ranges) to communicate different messages with their group members \cite{Holldobler1990}. Ants sitting near the source of this pheromone sense and respond quicker than others who wander in far distances. Thus both communication and sensing occurs within a small communication range. We have used this concept of communication range or locality in our LPCM. A suitable  range (or radius) of communication and sensing can be set at design time based on the capabilities of robots \citep{Agassounon+2002}. Alternately they can also be varied dynamically over time depending on the  cost of communication and sensing, e.g. density of peers, ambient noise in the communication channels, or even by aiming for maximizing information spread  \citep{Yoshida+2000}. In this study, we have followed the former approach. as our robots do not have the precise hardware to dynamically vary their communication and sensing ranges.
%%---------------------------------------------------------
\subsection{General characteristics of LPCM}
Our LPCM relies on the local P2P communications among robots. we have assumed that robots can communicate to its nearby peers within a certain communication radius, $r_{comm}$ and they can sense tasks within another radius $r_{task}$. They exchange communication signals reliably without any significant loss of information. A robot $R_1$ is a {\em peer} of robot $R_2$, if spatial distance between $R_1$ and $R_2$ is less than this $r_{comm}$.
Similarly, when a robot comes within this $r_{task}$ of a task, it can sense the status of this task. Although the communication and sensing  range can be different based on robot capabilities, we have considered them same for the sake of simplicity of our implementation.

Local communication can also give robots similar task information as in centralized communication. In this case, it is not necessary for each robot to communicate with every other robot to get information on all tasks. Since robots can random walk and explore the environment we assume that for a reasonably high robot-to-space density, all tasks will be known to all robots after an initial exploration period. In order to update the urgency of a task, robots can estimate the number of robots working on a task in two ways:  by either using their sensory perception (e.g. on-board camera) or  doing local P2P communication with others.

Similar to our centralized communication model, we can characterize our local communication model in terms of message content, communication frequency and target recipients \cite{Gerkey+2001}. Regarding the issue of message content, our local communication model is open. Robots can communicate with their peers with any kind of message. Our local model addresses the last two issues very specifically. Robots communicate only when they meet their peers within a certain communication radius ($r_{comm}$). Although in case of an environment where robots move relatively faster the peer relationships can also be changed dynamically. But this can be manipulated by setting the signal frequency and robot to space density to somewhat reasonably higher value.

In terms of target recipients, our model differs from a traditional publish/subscribe communication model by introducing the concept of dynamic subscription. In a traditional publish/subscribe communication model, subscription of messages happens prior to the actual message transmission. In that case prior knowledge about the subjects of a system is necessary \citep{Gerkey+2001}. But in our model this is not necessary as long as all robots uses a common addressing convention for naming their incoming signal channels.
%
%--------------------------
\subsection{Implementation}
In order to implement LPCM, our centralized communication scheme has been converted into a decentralized one where robots can use local observation and communicate with peers about tasks to estimate task-urgencies. Under this implementation, we present an emulation of this scenario where robots do not depend entirely on TPS for estimating task-urgencies, instead they get task information from TPS when they are very close to a task (inside $r_{task}$) or from local peers who know about a task via TPS.   The details implementation of LPCM and results can be found in \cite{Sarker2010thesis}.
%
%--------------------------------------
\subsection{Results and discussions}
For the sake of brevity, below we describe only the results from Series D experiments. Interested readers can find other results in \cite{Sarker2010thesis}. The sample raw task-urgencies of Series D experiments are shown in Fig. \ref{fig:raw-urgencies-SD}. In this case, we can see that an unattended task, \textit{Task4}, was not served by any robot for a long period and later it was picked up by some of the robots. 
%%%
%%%
The dynamic shop floor work-load is shown in Fig. \ref{fig:workload-SD}. This shows similar work-load as experienced in Series B experiments. 
%
%%----------------------------------------------------------------
% Plasticity -- SignalingFreq
\begin{figure}
\begin{minipage}[t]{0.48\linewidth}
\centering
\includegraphics[width=6cm, height=4cm]{images/SD-Local1m-Plasticity.eps}
\caption{\small Self-organized allocation of robots.}
\label{fig:plasticity-SD} 
\end{minipage} 
%%%
\hspace{0.5cm}
\begin{minipage}[t]{0.48\linewidth}
\centering
\includegraphics[width=6cm, height=4cm]{images/SD-Local-1m-SignalingFreqStat.eps}
\caption{\small Frequency of P2P signalling.}
\label{fig:local-signal-frequency-stat}
\end{minipage}
\end{figure}
%%%
The active worker ratios of Series D experiments are plotted in Fig. \ref{fig:plasticity-SC-SD}. Series C data shows us a large variation in this active worker ratios.
%%
From task-performance data of Series C we have got average production completion time 121 time-steps (605s) with SD = 36 time-steps (180s). For Series D,  average production completion time is 123 time-steps (615s) with SD = 40 time-steps (200s). According to Eq. \ref{eqn:min-pmm}, our theoretical minimum production completion time is 50 time-steps (250s) as discussed in Sec \ref{afm:results}.  The values of APCD are as follows. For Series C, $\zeta$ = 1.42 and for Series D, $\zeta$ = 1.46. For both series of experiments APCD values are very close.\\
%%
For APMW, Series C experiments give us an average time length of 359 time-steps (1795s). In this period we calculated APMW and it is 5 time-steps with SD = 17 time-steps and $\chi$ = 0.023420. For Series D experiments, from the average 357 time-steps (1575s) of maintenance activity of our robots per experiment run, we have got APMW, $\chi$ = 0.005359 which corresponds to the pending work of 2 time-steps (10s) where SD = 7 time-steps.
%%-------------------------------------------------
%%
%\begin{figure}
%\begin{minipage}[t]{0.48\linewidth}
%\centering
%\includegraphics[width=6cm, height=4cm, angle=0]{images/K-Group.eps}
%\caption{\small Task-specialization of robots.}
%\label{fig:K-Group} 
%\end{minipage} 
%\hspace{0.5cm}
%\begin{minipage}[t]{0.48\linewidth}
%\centering
%\includegraphics[width=6cm, height=4cm, angle=0]
%{images/Q-Group.eps}
%\caption{\small Time-steps to reach the peak values of task-specialization.}
%\label{fig:Q-G-SC-SD} 
%\end{minipage}
%\end{figure}
%%
By applying Eq. \ref{eqn:K-G} and Eq. \ref{eqn:Q-G} on our robots' task-sensitization statistics, we have got the peak task-sensitization $K^G_{avg}$ values: 0.39 (SD=0.17) and 0.27 (SD=0.10), and their respective time-step $Q^G_{avg}$ values: 13 (SD=7) and 11 (SD=5) time-step.
%-----------------------------------------------------------
\subsection{Comparisons}
Results from Series C and Series D experiments show us many similarities and differences with respect to the results of Series A and Series B experiments. Both Series C and Series D experiments show similar APCD values: 1.42 and 1.46 respectively, which are significantly less than Series B experiment result (APCD = 2.3) and are close to Series A experiment result (APCD = 1.22). This means that for large group, task-performance  is efficient under LSLC strategy (Series C and Series D) comparing with their GSNC counterpart (Series B).

Besides, in terms of task-specialization, the overall task-specialization of group in Series C ($K^G_{avg}$ = 0.4) is  closer to that of Series A experiments ($K^G_{avg}$ = 0.39) and interestingly, the value of  Series D ($K^G_{avg}$ = 0.27) is  much closer to that of Series B experiments ($K^G_{avg}$ = 0.30). So task-specialization in large group under LSLC strategy shows higher performance than their GSNC counter part. Besides task-specialization happens much faster under LSLC strategy as we can see that the average time to reach peak sensitization values  of the group,  $Q^G_{avg}$ in Series C is lower than that of Series A values by 25 time-steps.
\begin{table}
\begin{center}
\caption{Sum of translations of robots in Series A-D experiments.}
\begin{tabular}{|c|c|c|}
\hline \textbf{Series} & \textbf{Average translation (m)} & \textbf{SD} \\ 
\hline A & 2.631 & 0.804\\ 
\hline B & 13.882 & 3.099\\
\hline C & 4.907 & 1.678\\
\hline D & 4.854 & 1.592\\
\hline
\end{tabular}
\label{table:motion-cmp} 
\end{center}
\end{table}
%%
We have aggregated the changes in translation motion of all robots over time following an approach similar to task-urgency aggregation as described in Sec. \ref{validation:results} and summarized in Table \ref{table:motion-cmp}. From the robot motion profiles found in all four series of experiments, we have found that under LSLC strategy, robot translations have been reduced significantly. From this table we can see than Series C and Series D show about 2.8 times less translation than that in Series B experiments. The translation of 16 robots in Series C and Series D experiments are approximately double (1.89 times) than that of Series A experiments with 8 robots.  Thus the energy-efficiency under LSLC strategy seems to be higher  than that under GSNC strategy.

From the above results we can see that large group robots achieve better MRTA under LSLC strategy. The local sensing of tasks prevents them to attend a far-reaching task which may be more common under global sensing strategy. However, as we have seen in Fig. \ref{fig:raw-urgencies-SD}
some tasks can be left unattended for a long period of time due to the failure to discover it by any robot. For that reason we see that the values of APMW is slightly higher under LSLC strategy. But this trade-off is worth as LSLC strategy provides superior self-regulated MRTA in terms of task-performance, task-specialization and energy-efficiency.
%=======================================================================
\section{Conclusion}
This study has focused on reviewing bio-inspired communication strategies for self-regulated multi-robot systems with an emphasis on comparing two bio-inspired  communication and sensing strategies in producing self-regulated MRTA by AFM, an interdisciplinary model of task-allocation. Under the GSNC strategy, AFM has produced the desired self-regulated MRTA among a group of 8 and 16 robots. This gives us the evidence that AFM can successfully solve the MRTA issue of a complex multi-tasking environment like a manufacturing shop-floor. Under the LSLC strategy, AFM can also produce the desired self-regulated MRTA for 16 robots with different communication and sensing ranges.

From our comparative results, we can conclude that for large group of robots,  degradation in  task-performance and task-specialization of robots are likely to occur  under GSNC strategy that relies upon a centralized communication system. Thus GSNC strategy can give us better performance when the number of tasks and robots are relatively small. This confirms us the assertions made by some biologists that self-regulated DOL among small group of individuals can happen without any significant amount of local communications and interactions. However, our findings suggest that task-specialization can still be beneficial among the individuals of a small group which contradicts the claim that small groups only posses the generalist workers, but not the specialists.

On the other hand, LSLC strategy is more suitable for large group of individuals that are likely to be unable to perform global sensing and global communications with all individuals of the group. The design of communication and sensing range is still remained as a critical research issue. However, our results suggest that the idea of maximizing information gain is not appropriate under a stochastic task-allocation process, as more information causes more task-switching behaviours that lowers the level of task-specialization of the group. This might not be the case under a deterministic task-allocation scheme where more information leads to better and optimum allocations, but that is limited to a small group of individuals. Nevertheless, despite having the limited communication and sensing range, LSLC strategy helps to produce comparatively better task-allocation with increased task-specialization and significantly reduced motions or savings in energy e.g. battery power. 
%=======================================================================
\section{Future works}
This study can possibly be extended in co-operative task performance where different individuals with variety of task-skills can interact with each other directly. In our experiments, no dynamic task has been introduced during the run-time of our experiments. But AFM can be applied to a more challenging environment with suddenly appearing (and disappearing) dynamic tasks. Moreover, some more research can be done in order to figure out how to optimize the initial experimental parameters, e.g. robots' task learning and forgetting rate. Moreover, real implementation of communication range should be achieved by using suitable on-board communication module, e.g. Wifi or IR, with  relatively powerful robots. In our experiments, we have selected two fixed communication ranges that approximates LSLC strategy. However, much research is required to find optimum communication range as a property of self-regulation of individuals. %*************************************
\bibliography{intech-book} 
\end{document}
