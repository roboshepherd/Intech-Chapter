\documentclass[10pt,a4paper]{article}
\usepackage[latin1]{inputenc}
\usepackage{amsmath}
\usepackage{amsfonts}
\usepackage{amssymb}
\usepackage{makeidx}
%\author{Md Omar Faruque Sarker and Torbjorn S. Dahl}
\author{Md Omar Faruque Sarker and Torbj{\o}rn S. Dahl\\ %etc.
%}
%\authorrunning{Short form of author list} % if too long for running head
%\textit{
\small
Robotic Intelligence Lab, University of Wales, Newport\\
\small
Allt-yr-yn Campus, Allt-yr-yn Avenue, Newport, NP205DA, UK\\
\small
Mdomarfaruque.Sarker|Torbjorn.Dahl@newport.ac.uk
}
\title{Bio-inspired Direct Communication for Self-regulated Multi-robot System (MRS)}
\begin{document}
\maketitle
\textbf{Proposed Chapter Outline:}
\begin{enumerate}
\item Introduction
\item Atrractive Field Model - A Mechanism for Self-regulation in MRS
\item Communication in Biological Social Systems
\item Bio-inspired Direct Communication for MRS - A Case Study
\item Related Work
\item Conclusion
\newpage 
\textbf{\large Abstract}\\
This chapter intends to present a timely analysis on various direct communication strategies of multi-robot systems (MRS) along with our case study of implementing communication systems for  a large MRS with 40 robots. Both  centralized and local communication strategies are described in the context of self-regulated multi-robot task allocation (MRTA). Communication is the essence of any social interaction among individuals and their environments. In biological literature, animal communication has been broadly classified into two classes: direct or explicit communication and indirect or implicit communication (e.g., stigmergy of ants through pheromone trails). In this chapter, we intend to highlight mainly on direct communication with a limited coverage on indirect communication.

Since biological studies of communications of various social insects and other animals become one the major inspirations for designing scalable communication system for large MRS. Particularly, a branch of MRS called swarm robotics (SR) has been solely established by following the natural mechanisms of self-organizations in various social insects and other biological species. These self-regulated social systems provide us with living examples of scalable and flexible social communication systems  that effectively serve their various social needs.  In this article, we intend to review briefly on various biological communication systems in terms of communication modalities, information flow, group size, productivity and so forth.
 
In order to illustrate the suitable mechanisms for communication in MRS, we consider to put our study in the context of a self-regulated phenomena, i.e., division of labour that is also described as MRTA in MRS. As a part of a collaborative project, we have studied the behaviour of ants, humans and robots and have developed the attractive field model (AFM), a common formal model of division of labour in social systems. In this chapter we would like to present the validation of AFM that brings self-regulated task allocation in our large MRS. AFM relies on the presence of a system-wide continuous flow of information. Here our particular implementation shows that an interdisciplinary generic model of division of labour like AFM can not be realized in a MRS without an effective communication system. Our comparative results of self-regulated MRTA using both centralized and local communication system can illustrate this fact more vividly.

This book chapter is organized as follows. Section~2 presents attractive field model (AFM) that is used as a multi-robot task allocation (MTRA) mechanism. Section~3 reviews the various communication strategies found in biological systems. Section~4 describes our case study of bio-inspired direct communication model along with our implementation of MRTA including the interactions between the hardware, software and communication modules.  Section~5 discusses related background works and section~6 draws conclusions.
\end{enumerate}
\end{document}